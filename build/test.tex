%!TEX program = Traditional Builder with XeLaTeX
\documentclass[a4paper,12pt,oneside]{book} % 使文档按单面打印模式排版
\usepackage[UTF8]{ctex}
\usepackage{graphicx} % Required for inserting images
\usepackage{amsmath, amsthm, amssymb} % Math
\usepackage{wasysym, MnSymbol} % Greek alphabets
\usepackage{mathrsfs, amsfonts} % Math fonts
\usepackage{geometry} % Formatting
\usepackage{hyperref} % 引用链接
\usepackage{listings} % 代码输入
\usepackage{fancyhdr} % 引入fancyhdr宏包来控制页眉页脚
\usepackage{titlesec} % 定制章节标题样式
\usepackage{caption} % Optional, for caption customization
\usepackage{xcolor} % color
\usepackage{booktabs} % 使用booktabs表格形式
\usepackage{tabularx} % 在表格中制造换行
\usepackage{etoolbox} % 脚注通篇连续编号

% 脚注通篇连续编号
\counterwithout{footnote}{chapter} 

% 添加首行缩进,两个字符
\usepackage{indentfirst}
\setlength{\parindent}{2em}

% 设置页面边距
\geometry{top=3.5cm, bottom=2.5cm, left=2.5cm, right=2.5cm}
\setlength{\oddsidemargin}{0pt} % 奇数页左边距偏移
\setlength{\evensidemargin}{0pt} % 偶数页左边距偏移

% 避免出现所有作者名字
\usepackage[authoryear,etalmode=truncate,etaltext=it]{gbt7714}

% 设置页眉样式
\pagestyle{fancy}
\fancyhf{} % 清空默认页眉页脚
\fancyhead[C]{浙江财经大学硕士毕业论文}
% \fancyhead[C]{\makebox[\paperwidth][c]{浙江财经大学硕士毕业论文}} % 使用纸张宽度居中
% \fancyhead[C]{\rule{\paperwidth}{0.4pt} \\ 浙江财经大学硕士毕业论文}
\fancyfoot[C]{\thepage} % 居中显示页码
% \setlength{\headheight}{2.5cm}
% \setlength{\footskip}{2cm}


% 重新定义\chapter的样式,使编号和标题在同一行
\titleformat{\chapter}[hang] % 表示标题采用“悬挂”样式,即编号和标题在同一行。
  {\normalfont\huge\bfseries}{\thechapter}{1em}{}

% document headings
\title{基于个人动态最优居住地选择视角的劳动力回流行为考察}
\author{Suicidal Bumblebee}
\date{Jan 2025}


\begin{document}

\maketitle

\frontmatter
\renewcommand{\thepage}{\Roman{page}} % 将frontmatter页码改为大写罗马数字
% ---------------------------------------- 摘要 ----------------------------------------
\chapter{摘要}

\begin{center}
    \zihao{3}
    基于个人动态最优居住地选择视角的劳动力回流行为考察
\end{center}

随着经济的发展与社会工业化带来的社会关系剧变,劳动力迁移形成有规律的迁移模式是必然的。不同于我国研究中的传统城乡二元对立语境,本文基于理性预期构造了动态最优住址选择模型,基于CFPS数据对2010年至2022年间我国各省的人口流动进行实证检验,并得到的结论是(xxx、xxx、xxx)。本文的贡献在于(xxx、xxx、xxx)。

\textbf{关键词:} 动态迁移决策模型、劳动力迁移、收入引致、迁移摩擦


% ---------------------------------------- Abstract ----------------------------------------
\chapter{Abstract}

\begin{center}
    \zihao{4}
    \textbf{
    A Study of Return Migration Behavior Based on Dynamic Optimal Residential Location Decisions
    }
\end{center}

Here is the english version of abstract

\textbf{keywords}:a,b,c

% ---------------------------------------- 目录 ----------------------------------------
\thispagestyle{empty}
\tableofcontents

\mainmatter
% ---------------------------------------- 绪论 ----------------------------------------
\newpage
% \setcounter{page}{1} % 页码从此处开始记录
\chapter{引言}

纵观古今内外,“人往高处走,水往低处流”的规律总是屡试皆准。早在上世纪末,以\cite{krugmanIncreasingReturnsEconomic1991}和\cite{fujitaSpatialEconomyCities1999}为代表的新经济地理学就指出经济活动的集中会产生规模经济和网络效应,促成产业集聚。经济的聚集会导致区域不平衡发展,人们倾向于从“边缘”地区流向产业集聚、薪资较高和就业机会多的“中心”地区,形成“吸引效应”,这使得劳动力迁移有了移动的规律。欧美国家作为世界上人均GDP最高的区域,吸引大量来自其他相对落后国家的居民。根据世界银行(World Bank Group)和美国移民委员会(American Immigration Councile)公布的数据,
2023年美国的人均GDP为82769美元,外来人口达到了4780万,移民占美国人口的 14.3\%,比 1970 年的 4.7\% 增长了约三倍。相似地,在我国内部不同省份之间也发生着同样的人口迁移现象。改革开放以来,随着交通工具的进步和政策的放宽,劳动力自由流动在我国成为可能。解放的经济活力逐渐形成了“东富西穷、南富北穷”的局面。经济发展的不平衡不仅表现在地区收入差距上,也体现在人口分布的变化中。富裕地区吸引了大量来自相对贫困地区的劳动力。国家统计局2020年发布的《第七次人口普查》显示,流动人口达到3.76亿,占全国总人口比重分别为34.90\%和26.62\%,较2010年分别上涨88.52\%和69.72\%。
广东跨省流入高达2962.21万人,浙江也达到1618.65万人,上海跨省流入人口为1047.97万人。这三地的跨省流入人口数量位居前三。此外,北京流入841.8万人,位居第五。
同时,我国也存在多个人口输出大省\footnote{数据来源于2010年的《第六次人口普查》},例如
安徽省净向外输出约911万人,占本省户籍人口的13.29\%;
四川省净向外输出约956 万人,占本省户籍人口的10.63\%;
河南省净向外输出约约565万人,约占本省户籍人口的7\%。

在我国激烈的劳动力迁移浪潮中大致存在以下规律。
首先,劳动力净流入的区域符合常识中的迁移规律,劳动力从农村流向城市是迁移的主流趋势。大城市由于更高的工资水平、更丰富的就业资源、更高质量的基础服务设施,具有强大的“虹吸效应”,这一点与新经济地理学学者提出的观点相吻合。在可以预见的未来,这种劳动力向经济发达地区集中的趋势仍将持续。
其次,永久迁移与暂时性迁移之间存在显著差异。第七次人口普查显示,2020年时我国人户分离人口已达4.93亿,占总人口的34.16\%。(xxx)
并且,尽管总体上劳动力向发达地区流动,仍有部分劳动力“回流”到欠发达地区。这种反直觉的现象已被部分学者注意到(\cite{ShiZhiLeiJiaTingBingFuJiaTingJueCeYuNongCunQianYiLaoDongLiHuiLiu2012},\cite{RenYuanNongCunWaiChuLaoDongLiHuiLiuQianYiDeYingXiangYinSuHeHuiLiuXiaoYing2017}),表明某些群体在外迁后因各种原因选择返回原居住地,这揭示了迁移决策背后更为复杂的动机。\textit{DaVanzo (1983), who documented the richness  of individual migration histories, pointing out that although most individuals never move,  those who do are likely to move again, often returning to a home location. This means  that migration decisions should be viewed as a sequence of location choices, where the  individual knows that there will be opportunities to modify or reverse moves that do not  work out well.}


对于劳动力流动现象的研究,我国学术界长期围绕城乡二元分析框架与空间均衡模型展开。作为典型的发展中经济体,我国自计划经济时期形成的城乡二元结构构成了劳动力迁移的制度基础。改革开放后,工业化进程产生的劳动要素需求与农村剩余劳动力释放一拍即合,这一过程在学术研究领域直接映射为对传统二元经济理论的引用。其中,在\cite{lewisEconomicDevelopmentUnlimited1954}二元对立模型基础上,
\cite{todaroModelLaborMigration1969}通过引入失业率与预期收入,突破了无限劳动供给假设的刚性约束,其"即使存在失业风险,人口仍会因预期收入差距迁移"的核心命题,恰与中国城市化进程中农民工"候鸟式迁移"的特征契合,成为解释中国农民工流动现象的核心理论工具。而后,\cite{harrisMigrationUnemploymentDevelopment1970}进一步将城市正规部门与非正规部门纳入分析框架,通过工资刚性与就业概率的动态调整机制,构建起解释发展中国家城市失业与农村劳动力持续涌入并存现象的理论模型。这种强调制度分割与部门差异的分析视角,为中国学者解析户籍制度、土地制度等特殊约束条件下的劳动力流动提供了重要切入点。如\cite{XiongCaiYunNongMinGongChengShiDingJuZhuanYiJueCeYinSuDeTuiLaMoXingJiShiZhengFenXi2007}构建的农民工定居决策模型,通过引入城市拉力(就业机会、公共服务)与农村推力(土地保障弱化、收入差距)的交互作用,拓展了传统二元模型的解释维度。\cite{HuangZhongHuaNongCunTuDiZhiDuAnPaiShiFouZuAiNongMinGongShiMinHuaTuoDaLuoMoXingTuoZhanHeYiWuShiShiZhengFenXi2014}通过嵌入土地保险功能变量,\cite{ZhongShuiYingXiangChengRenKouLiuDongDeLiLunJieShiNongCunRenKouTuiChuShiJiaoTuoDaLuoMoXingDeZaiXiuZheng2015}纳入制度变迁因素。
部分文献虽然淡化了城乡之间的对立,但依然在均衡状态下分析工资和租金的确定,同时考虑移民流动对这些结果的影响。
\cite{ZongJiaFengDaChengShiZhiFuLiaoGengGaoDeGongZiMa2015}构建的三部门Rosen-Roback模型揭示,中国大城市存在显著工资溢价且技能异质性导致差异化集聚收益:高技能劳动力通过知识溢出获取短期增长红利,而低技能群体则需经历长期调整方能获益。
\cite{WangLiLiWoGuoRenKouQianYiChengBenChengShiGuiMoYuShengChanLu2020,WangLiLiTuDiGongGeiFangJieYuLaoDongLiKongJianPeiZhiXiaoLu2023,WangLiLiLaoDongLiLiuDongDuiChengShiGongZiYuFuLiDeYingXiangJiYuKongJianJunHengMoXingDeFenXi2024}通过将土地供给管制、迁移壁垒等制度变量内生化,定量识别出行政干预对劳动力空间配置的扭曲效应:建设用地指标的区域错配加剧东部大城市住房供给弹性不足,推高生活成本并阻碍生产率导向的人口集聚,导致2010年经济总产出损失达3\%-4\%。刘华仁(2024)则构建包含人力资本溢出效应的量化空间均衡模型,证明高技能劳动力的区域再配置可通过生产率与公共服务双重渠道提升社会福利,这为人才政策优化提供了理论依据。

尽管现有研究取得显著进展,其理论局限却愈发凸显,主要体现在三重维度:其一,动态演化机制缺位。主流文献多依赖静态或比较静态分析,将迁移决策简化为单期最优选择,忽视个体在生命周期内人力资本积累、预期调整与制度环境变迁的交互作用。例如,新生代农民工的"发展型迁移"倾向使其决策函数从生存工资最大化转向职业发展机会获取,这种偏好结构的代际演变亟需动态模型加以刻画。其二,微观基础建构薄弱。多数研究依赖宏观数据相关性分析,未能深入个体异质性(如风险偏好、社会网络、数字技能)对迁移路径分化的影响机制,导致政策仿真缺乏行为基础。其三,理论范式滞后于实践创新。传统模型难以解释"城-城流动"规模扩张、远程就业兴起等新业态对迁移地理的重构效应,尤其缺乏对迁移可逆性与多向性的理论建模。

上述局限导致既有研究难以回应两大核心命题:在城乡融合发展与数字技术渗透的复合冲击下,劳动力流动的长期均衡路径将呈现何种形态?制度变迁如何通过重构迁移成本收益函数影响个体的跨期决策?对此,Han(2018)虽尝试通过OLG模型捕捉迁移行为的代际动态,但其模型仍未能内生化制度约束与技能积累的交互效应。本文认为,突破现有理论桎梏需实现三个转向:从静态均衡转向动态演化分析,从宏观相关性转向微观行为基础构建,从单向城乡迁移转向多维度空间重构。为此,本研究构建一个动态迁移决策模型,引入制度摩擦、乡土黏性、数字技术渗透等异质性参数,构建包含预期调整与学习效应的动态迁移选择模型。通过反事实模拟与政策实验,旨在揭示以下机制:(1)户籍制度改革与土地确权如何通过改变预期稳定性影响迁移持久性;(2)人力资本积累速率差异如何引致迁移路径的分化;(3)远程就业普及如何重塑迁移的地理指向与频率特征。理论模型的构建不仅为解析中国劳动力流动的复杂图景提供新分析工具,亦能为推进以人为核心的新型城镇化战略提供政策启示。


本文的重要性在于,在当前工业化加速和经济转型的背景下,劳动力跨区域迁移问题愈发显得关键而复杂。尤其是在中国这一快速发展的国家,人口流动不仅直接影响城市化进程、区域经济平衡和资源配置效率,而且对社会福利和长期经济结构调整产生深远影响。深入理解和准确预测劳动力迁移趋势,不仅有助于把握经济内在发展动力,还能为政策制定提供科学依据,从而实现区域协调发展与社会整体进步。

本研究主要从收入差异这一视角切入,系统分析我国劳动力迁移的规律及其内在驱动机制。我们认为,收入差异作为影响劳动力跨区域流动的核心经济因素,其作用机理和效应值得深入探讨。通过构建理论模型和进行实证检验,本文旨在揭示: 不同区域之间收入水平的差距如何激发劳动力迁移;  收入差异在迁移决策中如何与就业机会、生活成本、制度摩擦等因素相互作用;  通过适当的经济政策干预,如何引导劳动力合理流动,从而优化资源配置、促进城市与农村的协调发展,并提升整体社会福利。
正如 Desmet 等人(\cite{desmetUrbanAccountingWelfare2013})所强调的:“Understanding the different forces that determine city sizes is crucial for answering a broad set of questions. What is the relative importance of these forces in determining the size distribution of cities? How much would we gain or lose if cities had similar amenities, technology levels, or frictions? How much reallocation would this cause? More generally, what are the welfare implications of the location of agents across cities?”——这一论述不仅指出了城市规模与人口重新配置对福利产生的深远影响,也启示我们:在面对不同区域之间的制度和经济差异时,探讨收入差异对于劳动力流动的影响具有特别重要的意义。如果各地区在公共服务、技术水平或制度摩擦方面趋于一致,劳动力配置和城市规模分布必将发生显著变化,进而影响整体社会福利。
通过本文的研究,我们期望为劳动力迁移问题提供一个从收入差异出发的理论与实证分析框架,进一步揭示影响迁移决策的内在经济机制,并为政策制定者在优化区域发展、促进劳动力合理流动和提升社会福利方面提供切实的理论支持和实证依据。

为实现上述研究目标,本文结构安排如下:  
第二章对国内外相关文献进行系统综述,梳理现有理论与实证研究成果,并明确前人工作的不足与本文的创新点;  
第三章构建理论模型,详细阐述模型的基本假设、机制设计及其与现实经济现象的关联;  
第四章针对实证模型中所需的似然函数进行详细推导,同时对主要变量作出必要的简化假设,以确保模型的可操作性和稳健性;  
第五章展示实证分析的主要结论,并通过对样本数据的不同截取和命题验证,探讨模型在解释我国劳动力迁移中的适用性;  
第六章总结全文,讨论研究局限,并提出未来研究可能的改进方向与应用领域。  
此外,更为详细的数学证明、程序代码和数据处理流程均附在文末的附录中。



% ---------------------------------------- 文献综述 ----------------------------------------
\chapter{文献综述}

\begin{center}
  \textit{
    In this part we begin by a brief introduction of spatial equilibrium model as comparison of dynamic optimal migration model. Then a short of history how previous DOMM worked, followed by an important introduction of dynamic discrete choice models. Later the application of DOMM. To expand DOMM to an empirical tool, a list of migratory factors is necessary at the end.
  }
\end{center}


\section{空间均衡方法}
本文采取的方法实则将迁移决策视作选择序列,在介绍这种方法之前,同样重要的是了解当下一种更受青睐的方法,即空间均衡方法。根据\cite{jiaEconomicsInternalMigration2023}的研究,通常有两种方式来研究劳动力迁移问题。第一种方法,如前所述,是空间均衡方法。


这种方法认为在迁移行为之后必然存在新的空间一般均衡,从而检视人口流动是如何影响各种变量的出清,例如本地工资、不同地区的房价等因素。


基础性工作有:

\cite{tieboutPureTheoryLocal1956}假设没有迁移成本且信息完全。消费者可以自由地在不同地点之间移动,并选择最能满足其偏好模式的社区;也就是说,他们“用脚投票”。

\cite{harrisMigrationUnemploymentDevelopment1970}提出了一个两部门的内部贸易模型,研究农村到城市的迁移,其中城市的最低工资高于农村收入。当城市部门的预期工资(即经失业调整后的实际工资)等于农村部门的平均工资时,达到均衡。

在\cite{robackWagesRentsQuality1982}的基本版本中,工人的间接效用取决于名义工资、住房成本和地方便利设施。工人是同质的且完全流动的。土地供应固定,住房供应弹性有限。该框架的一个重要含义是,任何对本地经济的冲击都会完全体现在土地价格中,从而不影响工人的效用。例如,正面的生产力冲击会提高名义工资和住房需求。工人因房价上涨而减少住房消费,从而允许一定程度的迁入。在新的均衡中,工人的福利并未改善,生产力的增加全部归土地所有者所有。

Glaeser和Gottlieb(2009)以及Moretti(2011)概述了一些重要例子。例如,Moretti(2011)提出了一个一般均衡模型,假设工人的流动性不一定是无限的,因为对某些地点的特殊偏好,住房供应也不一定是固定的。

\cite{diamondDeterminantsWelfareImplications2016}估计了一个结构性空间均衡模型,研究技能分类增加的原因及其福利影响。

例如,Coen-Pirani(2010)开发了一个动态一般均衡模型,强调个体异质性在迁移决策中的作用。

\cite{LiuHuaRenLiZiBenKongJianPeiZhiDeSheHuiFuLiXiaoYingYanJiuJiYuLiangHuaKongJianYiBanJunHengMoXingDeFenXi2024}

\cite{WangLiLiLaoDongLiLiuDongDuiChengShiGongZiYuFuLiDeYingXiangJiYuKongJianJunHengMoXingDeFenXi2024}

空间均衡方法在宏观情况下可能很有效,但在微观数据上却有困难,需要非均衡方法作为补充。



\section{个体选择模型与动态方法}
在简单介绍了空间均衡方法及其不适用的情况之后,我们来讨论第二种方法,即动态最优迁移模型(DOMM)。

70 年代末,Sjaastad(1962 年)将移民视为一项投资,强调了移民的动态方面——移民的预期成本和收益会随着时间的推移而变化。从生命周期的角度来看,个人(或家庭)决定是否以及何时移民。允许家庭做出多项移民决策大大增加了模型的复杂性。前期做出的决策(例如储蓄、教育、婚姻、生育)决定了当前时期的选择,对未来事件的预期也会影响当前的决策。从这个角度来看,移民和生育选择通过决策过程的“原始要素”联系在一起:当前机会(部分由过去的决策决定)、期望(预期的未来事件和结果)和偏好(分配给不同结果的值)。因此,经济视角提供了一个统一的框架,将几种重要的人口行为联系起来。

\cite{sjaastadCostsReturnsHuman1962} 提出了一个模型,假设个人根据迁移的成本和回报做出地点决策。回报包括货币回报以及基于位置偏好的非货币回报。

\cite{mincerFamilyMigrationDecisions1978} 开发了一个模型来允许家庭联合移民决策,认识到家庭的最佳地点选择可能与个人配偶的最佳地点选择不同。

在接下来的 40 年里,这些模型在许多重要维度上得到了扩展,包括效用函数、时间范围和位置选择集等因素。2011 年著名的 Kennan 和 Walker 个人移民选择模型结合了许多这些进步;以下我们将该模型称为 KW。KW 中的结构动态模型允许个人进行丰富的决策过程。它允许个人技能和位置偏好的异质性,以及迁移成本的异质性。它还允许许多备选位置选择,每个位置的收益流通常不确定,允许工资和偏好的位置匹配成分,允许连续的位置决策而不是单一选择。该模型将个人移民决策作为最佳搜索过程来解决,旨在描述工人对不同地点工资差异的局部均衡反应,假设个人最大化预期终生收入,扣除迁移成本。

the basic differences are shown in the table \ref{tab:空间均衡方法与个体选择方法的对比}.

\begin{table}[!ht]
\centering
\begin{tabularx}{\textwidth}{@{}cXX@{}} 
\toprule
\textbf{比较维度} & \multicolumn{1}{c}{\textbf{空间均衡方法}} & \multicolumn{1}{c}{\textbf{最优居住地选择模型}} \\ \midrule
理论基础 & 基于新古典经济学一般均衡理论,强调区域间要素的联动调整、均衡状态和市场出清 & 基于效用最大化理论,强调个体决策和行为微观基础,是局部均衡视角下的产物 \\
代表模型 & Tiebout模型、Rosen-Roback模型 & 动态离散选择模型、Logit计量模型、推拉模型 \\
研究视角 & 关注城市/区域层面的整体均衡状态 & 聚焦个体/家庭的最优决策过程 \\
核心假设 & 完全流动性、完全信息 & 效用最大化、异质性偏好 \\
时间维度 & 多为静态或比较静态分析,关注长期均衡 & 可以进行动态分析,关注短期和长期迁移决策 \\
均衡作用 & 侧重于个体和企业在不同地点之间无差异的空间均衡条件 & 均衡是个体选择的结果,但重点在于决策过程本身 \\
空间维度 & 侧重地区间的宏观特征差异 & 同时考虑地区特征和个体-地区匹配特征 \\
异质性处理 & 弱异质性,通常假设同质性劳动力或有限异质性 & 强异质性,明确建模个体异质性和偏好多样性 \\
内生性处理 & 通过均衡条件内生确定工资、房价等 & 通过计量方法处理迁移决策中的内生性问题 \\
计量方法 & 结构方程估计、联立方程系统 & 离散选择模型、条件Logit、嵌套Logit、混合Logit \\
政策应用 & 适合评估宏观政策对要素价格和人口分布的影响 & 可模拟个体行为响应(如迁移补贴对不同收入群体的激励差异) \\
数据需求 & 依赖区域层面汇总数据(平均工资、房价、失业率) & 需要微观个体数据(人口普查、家庭追踪调查) \\
处理机制 & 强调价格机制的调节作用 & 强调个体选择机制和选择集构建 \\
预测能力 & 预测区域间总体流动趋势 & 预测个体迁移行为和选择概率 \\
福利分析 & 基于总体均衡状态的福利评估 & 基于个体效用变化的福利评估 \\
典型应用 & 区域经济收敛性研究,税收与公共服务匹配分析 & 农民工市民化决策预测,高技能人才跨国流动研究 \\
局限性 & 忽略个体决策异质性 & 微观模型难以反映系统均衡效应 \\ 
\bottomrule
\end{tabularx}
\caption{空间均衡方法与个体选择方法的对比}
\label{tab:空间均衡方法与个体选择方法的对比}
\end{table}


进入21世纪,DOMM出现了新的趋势。结合动态离散模型,DOMM找到了新的命运。
要理解这个现代版本的DOMM,
我们应该知道它的根源在于动态离散选择(Dynamic discrete choice, DDC)理论,该理论遵循单个决策主体动态的研究模式。


离散选择模型的关键特征是:

1,个人面临多个、有限、离散、完整和可分离的选择。

2,无论选择与否,选择都不会被淘汰。

3,不确定性

4,显示偏好和微观数据


最基础的离散模型开发了随机效用模型来描述选择之间的决策。基于RUM中误差项的关键假设,创建了logit模型。


该领域的杰出作品有:

- 最初 Thurstone (1927) 从心理激励的角度 引入了比较判定定律 (a Law of Comparative Judgment)。定律 指出具有真实激励水平的选择项$i$被感知含有一个正态误差:$V_i+\epsilon_i$。

- Marschak (1960) 将这种感知到的刺激$V_i+\epsilon_i$解释为效用 并对包含随机因素的效用最大化的选择概 率进行了理论上的推导。Marschak 将其称为随机效用最大化 (Random Utility Maximization, RUM) 模型。

- 起初 logit 公式是 Luce (1959) 通过对选择概率的特性进行假设推导而得 他引入了 被称为 “不相关选项间的独立性” (Independence from Irrelevant Alternatives, IIA) 特性 (或公理) 通过允许从二元选择的实验中推断多元选择概率来简化实验选择数据的收集。Luce (1959) 告诉我们当选择项 i 在选择集合$C$中 IIA 特性意 味着精确的效用 (strict utilities)。

- Marschak (1960) 揭示了这一公理意味着对于一个有限的范围而言 IIA 特性就意味着 RUM。

- logit 公式与不可观测效用之间的联系是由 Marley 发展起来的,他揭示极值分布导致了 logit 公式的产生。

- McFadden (1974) 从反面论证用于选择概率的 logit 公式必然暗含着不可观测效用服从极值分布,并从计量经济学的视角研究了 Luce 的模型 并将精确的效用设定为选择项可观测变量 的函数。决策者 n 选择选项 i 的概率写成一种简洁的封闭型的表达形式:

- 随后GEV,probit,mixed logit等多种办法被研发突破传统logit的局限性





随着学术界的发展,将离散选择应用于动态框架的需求比以往任何时候都更加重要。在反复的召唤下,二十世纪八十年代开始,对动态版本离散选择模型的研究开始慢慢展开。

该领域的开创性论文包括:

- DDC 模型的历史深深植根于动态规划的发展,这是 Richard Bellman 于 1957 年在其著作《动态规划》(Dynamic Programming)中提出的一个概念。Bellman 的工作为解决顺序决策问题提供了数学框架,这成为马尔可夫决策过程 (MDP) 的基础。 

David Blackwell 于 1962 年发表的论文《离散动态规划》(《数理统计年鉴》,33,719-726)进一步推进了这一研究,为 MDP 建立了理论基础,尤其是在离散环境中,这是 DDC 模型的核心。

到 20 世纪 80 年代初,MDP 已在微观和宏观经济理论、金融和运筹学中得到广泛应用,正如 John Rust 于 1994 年发表的调查《马尔可夫决策过程的结构估计》(《计量经济学手册》,第 4 卷,第 51 章,3081-3143,第 51 章马尔可夫决策过程的结构估计)中所述。这一时期,MDP 的应用包括最优库存政策(Arrow 等人,1951 年)、不确定性下的投资(Lucas 和 Prescott,1971 年)和不确定性下的最佳增长(Brock 和 Mirman,1972 年;Leland,1974 年)等。这些研究虽然并非 DDC 的独有,但为将动态规划应用于经济学中的离散选择问题奠定了基础。

经济学中的 DDC 模型领域开始于 20 世纪 80 年代中期,\cite{pakesPatentsOptionsEstimates1986}将专利续展决策建模为动态离散选择问题,使用随机控制模型估计持有专利的价值,这突出了模型在处理不确定性和离散选择方面的效用。

\cite{rustOptimalReplacementGMC1987}经常被引用为开创性著作。它提供了经济学中最早的实证应用之一,为公交公司维护主管 Harold Zurcher 是否更换巴士发动机的决策过程建模。这项研究使用了最大似然估计并引入了嵌套不动点 (NFXP) 算法,成为该领域的经典示例。

- 后来,\cite{hotzConditionalChoiceProbabilities1993}引入了条件选择概率方法。这种方法避免了为每个参数猜测求解完整动态规划问题的需要,大大提高了计算可行性,并成为一种领先的非解估计方法。

Zvi Eckstein 和 Kenneth I. Wolpin 于 1989 年发表的论文《动态随机离散选择模型的规范和估计:一项调查》(《人力资源杂志》,24(4),562-598,《动态随机模型调查》)和 Rust 于 1994 年发表的论文提供了全面的概述,更新了该领域并强调了其发展。这些调查注意到静态离散选择模型的自然延伸,其应用扩展到劳动经济学、产业组织等领域。




- Miller 1984建立 了一个关于工作匹配的模型研究了工作类型和职业趋向对体现具体工作价值的差异性,其实证表 明前期的职业选择会影响对后期职业的抉择。

- Stock和Wise1990在他们早期的一篇论文中 用选择模型来模拟  在一个公司的退休金计划变化对于退休的影响与在社会安全供给变化对退休 的影响。

- 也有学者将此研 究领域扩展到生育研究,Wolpin于1984年构造 了一个关于生育率和儿童死亡率的动态随机模型 。

- 更有学者将此应用于犯罪研究,Imai和Krishna 2004采用月度面板数据和极大似然方法构造和估计了犯罪行为动态模型。

- \cite{keaneEmpiricalApplicationsDiscrete2009},\cite{aguirregabiriaDynamicDiscreteChoice2010}给出了一个完整的介绍





DDC 的基本特征如下:

1、离散时间序列中的动态性。

2、个体面临多个、有限、离散、完整且可分离的选择。

3、无论选择与否,选择都不会被淘汰。

4、马尔可夫决策过程。这意味着仅有T-1期的状态变量影响T期,而不是更前期。

5、不确定性,或者说随机决策环境

6、显示偏好和微观数据

7、单个agent



毫无疑问,它是研究动态迁移决策过程最合适的工具。动态最优迁移模型(DOMM)当然是 DDC 模型的一部分,或者至少与之有着深厚的联系。通过表格\ref{tab:DOMM和DDC假设的对比}我们可以看到相似之处:

\begin{table}

\caption{动态最优迁移模型和动态离散选择模型之间的关键性假设的对比}
\label{tab:DOMM和DDC假设的对比}  
\end{table}

- 在离散时间序列的每个时期做出的动态迁移选择

- 迁移目的地是广阔的、有限的、离散的和可分离的

- 代理人可以是个人或家庭

- 由于在整个时间序列中始终存在选择,因此 DOM 描述了可撤回的迁移行为。

- it took Random Utility Model (RUM) as base in utility theory, allowing for uncertainty随机效用最大化模型(Random Utility Maximization, RUM),这一类模型假设个体的居住地选择是基于效用最大化。每个居住地对应一个效用函数,个体通过权衡不同地区的效用,选择效用最大的地区。



国内对于DDC的研究较少
多数集中在对于机制的探讨
例如\cite{QinSongDongTaiChiSanXuanZeMoXingJiQiGuJiFangFaFaZhanZongShu2010}和\cite{MaYaQiDongTaiChiSanXuanZeMoXingDeMoNiFenXi2016}对于DDC的原理和方法给出了介绍和讨论。

最优居住地序列选择模型的关系如图\ref{fig:最优居住地序列选择问题的理论来源venn diagram}所示。

\begin{figure}[!ht]
\centering
\includegraphics[width=0.8\textwidth]{images/optimal_residential_sequence.drawio.pdf}
\caption{最优居住地序列选择问题的理论来源}
\label{fig:最优居住地序列选择问题的理论来源venn diagram}
\end{figure}

通常而言有两种方法用于构建动态离散选择模型。

考虑时间背景的模型常 被称为动态选择模型(dynamic choice model),其典 型的两种处理方式如下。 第一种是滞后因变量的运用。一些研究参照 时间序列分析方法,将上一期或多期的选择结果作 为本期解释变量纳入模型,试图追踪并解释同一类 行为随时间变化的规律。例如
,Smith(2005)在利用 混合 Logit 模型研究渔民 10 年间的捕渔地选择时, 将过去所有的选择结果全部纳入,结果表明自变量 的瞬时变化具有不同的近期和远期效应;Dargay 等 (2007)在对居民小汽车保有水平和交通模式选择进 行跟踪研究时,发现将滞后因变量分别纳入定序 Logit 和二项 Logit 模型均取得良好的解释力;Chatterjee(2011)在使用定序 Logit 模型预测公共汽车的 使用频率时,发现滞后因变量的加入显著提高了预 测精度;Grigolon 等(2014)建立了动态混合 Logit 模 型以研究 7 年间人们对不同长度假期的选择,通过 将同年、去年、两年前的选择结果纳入模型,发现与 中短假期相比,长假的选择更易受前期选择结果的 影响。

第 二 种 是 期 望 贴 现 效 用 (expected discounted utility)的运用。如果说前类研究主要关注过去,则 此类研究更多着眼于未来,它试图反映个体在进行 选择决策时,会考虑到当前结果将影响未来的选择 情景,因此以更长远的眼光选择能使总体期望贴现 效用最大化的选项。例如,Attanasio 等(2012)在研 究儿童上学的选择中,除纳入上学、工作两个选项 的当前效用外,还考虑了上学投资对未来工作的预期回报及其不确定性;McKee(2006)用类似方法研 究了中老年人的退休选择决策,其中考虑了基于未 来收入、开支、健康状况等因素的期望效用的影响; Todd 等(2010)将此类方法统称为动态规划离散选 择 模 型 (discrete choice dynamic programming model, DCDP),并总结了其在教育、经济、移民等政策评 价中的应用;Lapparent 等(2012)在研究居民是否保 有车辆的选择时,发现其重要决策因素之一是对未 来车辆使用的预期以及由此折算的当前需求,指出 将未来考虑纳入模型对理解决策过程具有重要意义。

\cite{kennanEffectExpectedIncome2011} used a DDC-based way to study migratory phenomenom in US

Using DDC to work with migration problem is at its core a method of non-equilibrium study (\cite{jiaEconomicsInternalMigration2023}). 
It is 用于分析个体或家庭如何在多个备选居住地之间做出最优选择的模型,考虑各种经济、社会、地理和政策因素,以解释人口迁移、城市发展和住房市场的动态。其背后的基本思想是,当搬迁的净预期收益超过成本时,个人会选择搬迁假设迁移是对外部冲击或永久性工资增长机会的响应,从而提高个体的效用水平。Such ideas were seen as early as in the work of Greenwood (1975)。This enable the users to apply individual-level heterogeneity.

深化迁移成本异质性研究
\cite{bayerDynamicsInterstateMigration2012}构建微观经济结构模型处理迁移激励的不可观测性与强自相关问题,其聚合分析显示:考虑动态自选择效应后,跨州迁移成本不足家庭年收入中位数的一半。


DOMM 可以作为return migration研究的工具。或者说,回归移民将 DOM 扩展为可重复的离散选择模型。而之前关于撤回移民的论文则从更静态的角度出发,例如\cite{RenYuanNongCunWaiChuLaoDongLiHuiLiuQianYiDeYingXiangYinSuHeHuiLiuXiaoYing2017}, \cite{ShiZhiLeiJiaTingBingFuJiaTingJueCeYuNongCunQianYiLaoDongLiHuiLiu2012}.

该领域的基础工作包括

- Direx 1988

- Kennan and Walker 2010, 2011

- Thom 2010 提供了更现代版的分析

- \cite{dustmannEconomicsTemporaryMigrations2016}聚焦国际临时迁移的研究延续这一动态特征,但其模型进一步探讨技能积累差异、技能回报率区位差异及本土/东道国消费偏好如何驱动跨国往返迁移。研究表明,即使不存在KW模型中的外生冲击,上述机制仍可引致回流行为。


而国内对于return migration的研究往往停留在静态选择的阶段,采用简单的logit回归。
例如\cite{RenYuanNongCunWaiChuLaoDongLiHuiLiuQianYiDeYingXiangYinSuHeHuiLiuXiaoYing2017}


推拉理论从成本—收益的角度研究认为,人口流动的目的在于改善其自身的生活条件,当流入到城市的农村劳动力在城市  中的生活条件并没有得到改善( Murphy,2002) ,或者迁移者家乡有更  好的投资机会( Christiansen \& Kidd,1983) 时,他们往往就需要再次进行选择。
这与我国目前研究中涌现的回流问题不谋而合
劳动力回流现象和可撤回迁移行为在本质上相同
所以与可撤回迁移比较相关的研究一般来自于对于劳动力回流现象的研究


\cite{ShiZhiLeiJiaTingBingFuJiaTingJueCeYuNongCunQianYiLaoDongLiHuiLiu2012}得出结论“本文根据湖北和河南两省的农户抽样调查数据,建立农村迁移劳动力回流决策的影响因素模型,从家庭决策的视角分析了家庭禀赋对迁移劳动力回流的影响及其作用机制。数据分析结果表明:家庭人力资本越丰富,劳动力越容易选择留在农村就业或者回流农村,但是家庭人力资本值达到一定程度后,农村劳动力又倾向于外出就业。家庭社会资本有助于迁移劳动力外出务工,但是随着家庭社会资本值的增加,那些家庭社会资本更为丰富的家庭的劳动力则更愿意回流家乡就业。丰富的家庭经济资本同时可以产生收入效应和替代效应,家庭经济资本可以为外出务工提供物质支持,但是丰富的家庭经济资本又会促使迁移劳动力回流农村,总体来说后者更为明显。”






DOMM can care about family decisions

- \cite{gemiciFamilyMigrationLabor2007}, 解析生命周期与家庭因素, 构建包含家庭内部协商机制的动态迁移模型,校准结果显示:家庭纽带会抑制人口流动性、减缓工资增长并降低家庭稳定性。此类分析弥补了单期模型无法捕捉生命周期关联效应的缺陷。







- 构建政策评估工具。此类模型凭借简约参数结构与良好可操作性,成为反事实分析的理想框架。例如,基于KW框架的扩展研究已成功量化跨州福利待遇差异与州政府高等教育补贴对迁移决策的影响(Kennan \& Walker 2010; Kennan 2020)。













\section{影响劳动力迁移的因素}

\textbf{To expand the DOM model into a more useful empirical tool, another range of papers discussing migratory factors should be listed.}

DOM模型可以在随机效用函数的确定性部分中嵌入多种因素从而考察他们对迁移行为的影响。类似的想法在推拉模型以及其他许多计量经济学论文中都有体现。推拉模型考虑了迁移过程中的各种因素。它是研究迁移决策原因的典型方法。

Tiebout模型(“用脚投票”理论): Tiebout 模型是最早描述个体在多个区域之间迁移选择的理论。该模型假设个体根据不同地区提供的公共产品和税收水平来选择居住地,像是在“用脚投票”。如果一个地区提供的公共服务较好、税负较低,个体会倾向于选择该地区。

\cite{leeTheoryMigration1966}\textit{在推拉理论的基础上加入中间障碍因素和个人因素。同时,他认为流入地和流出  地各自都有推和拉两种力量。中间障碍因素主要包括制度安排、距离远近、文化差异等,在中国尤  其表现为户籍制度的影响。}

Bagne 1969推出了推拉模型,系统阐述推拉理论,指出流入地的有利生活条件为拉力,流出地的不利生活条件为推力,劳动力流动受这两股力量的影响。可以将其视作计量经济学线形多元回归,通过添加多种推拉因素作为控制变量,消除


To specification, factors including but not limited to as follow:

- income improvement. Higher income by comparison induces migration. 

- house price. Most researches agree the effect coulde be U-shaped.

- public service is a very large genre, we try to avoid such big word. It is a combination of all kinds of government-provided goods and natural resources. And the richer diveristy and better quality local government can provide, more migration is induced.

- weather

- hazzards

- cultural barrier is another big word genre, including but not limited to language barrier, habits, identity recognition and 

- direct migration cost, a total amount of present money paid for the transfer from one place to another. 迁移成本模型: 迁移成本模型考虑了个体在选择居住地时需要支付的成本(如搬家费用、适应新环境的成本等)。在考虑这些成本的情况下,个体可能会选择效用次优的居住地以避免高迁移成本。



\textbf{房价}

中国居高临下依旧日益增长的房价是讨论影响劳动力流动时无法避免的话题。

1.\cite{WangLiLiTuDiGongGeiFangJieYuLaoDongLiKongJianPeiZhiXiaoLu2023}认为房价的高涨阻止了劳动力的迁移,\textit{房价高涨阻碍人口向 高生产率的东部大 城 市 转 移, 加 剧 劳 动 力 资 源 的 空 间 错 配, 导 致 2010 年 经 济 总产出损失3\%—4\%。放松东部大城市的土地利用规制, 有利于改善劳动力配 置效率。}
2.Helpman (1998)在Krugman 1991的基础上引入住房市场因素,指出房价会影响劳动者相对效用,从而抑制劳动力在高房价地区的集聚;还提出经济集聚导致的劳动力涌入也会推高房价。Brakman et al. (2002); Saks (2004); Rabe \& Taylor (2012)实证检验了Helpman的理论。Hanson et al. (1999, 2005)实证分析支持Helpman(1998)的结论。

房价的倒U影响:
3.Dohmen (2005); Meen \& Nygaard (2010)指出尽管高房价地区会抑制劳动力流入,但预期套利机会会促使劳动力流入。
4.\cite{ZhangLiFangJieRuHeYingXiangLaoDongLiLiuDong2017}\textit{在理论上,本文论证了房价的拉力作用和阻力作用,一 方面是由于房价作为备择城市的城市特征信号降低了预期未来收入的不确定性所带来的 拉力,另一方面是房价作为居住成本压缩可支配收入所产生的阻力,两种作用最终对劳动 力流动产生先吸引后抑制的倒 U 型影响。本文认为,城市的高房价对劳动力流动同时存在正反两方面的拉力和阻力。一方面,高房价意  味着城市更好的发展前景、个人更匹配的工作机会和更大的财富增长空间,同时还意味着更优质的公共服务和基础设施等,因此高房价能够吸引人才流入; 另一方面,快速上涨的房价极大地加重了外来劳动力的生活成本,一定程度上也阻碍了城市引进人才和创新创业。}同时,这篇文章给出了一个重要的启示\textit{房价并不仅是居住成本, 而且还是流入城市特征的信号,减少预期收入的不确定性}。
周颖刚2019经济研究\cite{ZhouYingGangGaoFangJieJiChuLiaoShuiJiYuZhongGuoLiuDongRenKouDeWeiGuanShiJiao2019}
\textit{根据Roback(1982)和Diamond(2016),一方面劳动力从低工资地区向高工资地区流动,以提高劳动力家庭的效用水平;另一方面,房价是劳动力在城市居住的主要成本,高房价水平降低了劳动力在居住城市的效用水平。}
Diamond 2016\cite{diamondDeterminantsWelfareImplications2016} \textit{Relatedly, highly educated Americans have become increasingly concentrated in larger cities (Diamond 2016).}



\textbf{公共服务与福利}

在社会学的研究中,舒适度这个概念是人口迁移的主要因素,这一概念涵盖了空气质量、公共服务、城市天气等多种变量。以王丽娜2007为例

\textit{根据 Rosen-Roback 的城市空间均衡理论,劳动力的空 间流动受到收入、房租( 生活成本) 以及城市宜居性特征( amenity) 的影响。}

Tiebout 1956

夏怡然2015,城市间的“孟母三迁”——公共服务影响劳动力流向的经验研究
\cite{XiaYiRanChengShiJianDeMengMuSanQianGongGongFuWuYingXiangLaoDongLiLiuXiangDeJingYanYanJiu2015}

\cite{SunWeiZengKongQiWuRanYuLaoDongLiDeKongJianLiuDongJiYuLiuDongRenKouJiuYeXuanZhiXingWeiDeYanJiu2019}
基于Rosen-Roback模型得出空气污染对于流动人口的就业选址具有显著的负向影响

Banzhaf \& Walsh( 2008) 研究发现人们确实会对环境质量“用脚投票”,并且 同时存在规模效应和结构效应———污染物排放增加的地区的人口下降了 5\% —9\% ,而污染物排放 减少的地区则经历了 5\% —7\% 的人口增长; 周边污染物排放增加的社区的平均收入水平发生了下 降,这反映了富裕家庭的迁出或贫穷家庭的迁入。


\textit{Dorfman (2016): This paper extends the literature on amenity migration by focusing on healthcare access for later‐life migrants.}


福利金引致的劳动力迁移行为
\cite{benjaminImportingPoorWelfare2004}
\cite{mckinnishWelfareinducedMigrationState2007}
随着美国的高强度非法移民现象而兴起的研究

Egglestondd等研究了新农村医保给老年人提供了更多医疗保障,缓解了年轻人的迁移成本压力,增强了年轻人的迁移意愿




\textbf{户口}

与户籍挂钩的公共福利政策阻碍了我国的城市化进程(Ngaietal。,2018)。\cite{ngaiChinasMobilityBarriers2019}: "China‘s hukou system imposes two main barriers to population movements. Agricultural workers get land to cultivate but are unable to trade it in a frictionless market. Social transfers (education, health, etc.) are conditional on holding a local hukou. We show that the land policy leads to over-employment in agriculture and it is the more important barrier to industrialization. E§ective land tenure guarantees and a perfect competitive rental market would correct this ine¢ ciency. The local restrictions on social transfers favour rural enterprises over urban employment with a relatively smaller impact on industrialization.

李强2003认为由于户籍制度,中国的人口流动将不再遵循一般的推拉规律

\cite{LuYiLongHuKouHuanQiZuoYongMaHuJiZhiDuYuSheHuiFenCengHeLiuDong2008}用OLS分析户籍的社会流动影响

在我国的户籍政策下,流动人口因无法享受居住地的医疗、教育、养老等公共福利,仍需支付很高的迁移成本(Tombe and Zhu,2019)。

\cite{ZhouWenTuDiLiuZhuanHuJiZhiDuGaiGeYuZhongGuoChengShiHuaLiLunYuMoNi2017}
在刘易斯二元经济模型基础上放松了农村剩余劳动力无  限供给和劳动力同质性假设,同时考虑利益集团的博弈和移民在城市的卢卡斯人力资本外部性,通过引入土地流转和人口迁移的限制,进而刻画出土地制度和户籍制度改革对城市化水平、要素价格  及居民福利的影响;
土地控制和户籍控制对于劳动力流动的影响。

\cite{WangLiLiWoGuoRenKouQianYiChengBenChengShiGuiMoYuShengChanLu2020}将户口视作人口流动摩擦的来源,使用包含人口流动摩擦的空间均衡模型研究发现,我国在2000 年的流动人口存量能够解释11.1\% 的 总 产 出 增 长; 从  2000年到2005年,我国 劳 动 力 的 平 均 迁 移 成 本 下 降 了 约 14.6\%, 使 经 济 总  生 产 效 率 提 高 4\% 。在经济新常态下,城市化成为支撑我国经济发展的重要领域。然而,户  籍政策带来的人口流动壁垒尚未消除,我国各城市的劳动力进入壁垒高于农  村非农业部门,同时大城市的劳动力进入壁垒高于中小城市,造成我国城市  化滞后于工业化,城市规模不足,制约着我国经济的持续发展。因此,有必  要进一步改革我国的户籍政策,逐步放松大城市的户籍限制。进一步降低人口流动壁垒将有利于我国城市规模的扩张与劳动力资源配置效率的改进。


\textbf{年龄}

年龄$a$是讨论个体迁移问题中很重要的一环
年轻人缺少家庭因素更容易定居在外地
年轻人对于未来容易抱有较高幻想,即倾向于认为自己有更高的预期收入

标准的人力资本理论解释是,移民是一种投资:如果其他地方有更高的收入来源,那么越早移民,就越早获得收益。此外,由于工作寿命有限,对年轻工人来说值得的移民对年长工人来说可能不值得,因为更高的收入来源没有时间抵消移民成本(Sjaastad 1962)。换句话说,移民更有可能是年轻人,原因与学生更有可能是年轻人相同。

\cite{wozniakAreCollegeGraduates2010}讨论了年轻人更倾向于进行迁移,\cite{molloyInternalMigrationUnited2011}也是如此



\textbf{人力资本}

Moretti (2004); Fu and Liao (2012)指出人口密度高的地方有利于技能匹配和获得学习机会,劳动力更倾向于流向教育水平高、人口密度强的城市,这实际也是侧面证明了预期收入

刘毓芸等(2015)\cite{LiuYuYunLaoDongLiKuaFangYanLiuDongDeDaoUXingMoShi2015}:基于文化经济学理论,指出方言距离较小时促进劳动力流动,反之则阻碍劳动力流动。

邢春冰研究发现相比较学历低的劳动力,学历高、经验丰富的劳动力更倾向于迁移

\textit{Schultz (1961), who considered migration as a form of investment in human capital}

Diamond-Mortensen-Pissarides 模型[14]突破了传统经济学的劳动者与企业之间无摩擦就业的限制,将搜寻摩擦纳入模型来分析均衡工资问题,结论是搜寻摩擦与技能水平对工资有显著决定作用。Rama[15],Petrongolo等[11]研究表明学历越高,工作技能专一性更强,就业匹配摩擦越大,失业率越高。

Malamud 和 Wozniak (2009) 将征兵风险作为教育的一个衡量工具,发现教育水平的提高会导致移民率的上升(另一种情况是,上大学的人迁移成本较低,因此即使他们没有上过大学,他们的移民率也会更高)。


\textbf{语言}

\cite{LiuYuYunLaoDongLiKuaFangYanLiuDongDeDaoUXingMoShi2015}假定方言距离使得暂时性收入的方差变大:\textit{Pendakur\&Pendakur(2002) 认为语言是民族认同和民族身份的一个重要维度,他们研究了加拿大三大城市区中13种少数语言的经济回报,控制主体语言后,掌握少数语言的个体往往是低收入,并将其归因于少数语言的低认同所带来的负面影响。需要强调的是,本文不是考察流动者本人所讲的方言,而是流动者所在的流入地和流出地的方言。与现有文献相比,本文的识别策略排除了个体掌握语言所体现的人力资本因素,至少在本文看来,能更干净地识别语言的影响。本文与Chenetal。(2014) 的工作最为接近。他们研究了流人上海的劳动力所掌握的上海方言对其收入的影响,并用劳动力的出生地与吴方言的方言距离作为工具变量。本文则采用全国样本,考察不同地区的方言对劳动力流动的影响,揭示劳动力跨方言流动的模式。}

\textit{另一方面,与认同论所强调的一致(Pendakur\&Pendakur,2002;McPhersonetal。,2001),方言作为身份认同的重要标志,使个体更加认同与自己所讲方言相近的城市,排斥与自己所讲方言差别较大的城市,因为这种差别会增大在流入地发生被偷、骗、抢等意外的风险。}





\textbf{不完全信息下的偏好信息——知名度}






研究劳动力迁移影响因素的文献如表格\ref{tab:影响劳动力迁移的因素}所归纳。

\begin{table}[!ht]
\centering
\begin{tabularx}{\textwidth}{@{}ccX@{}}
\toprule
\textbf{影响因素} &  & \multicolumn{1}{c}{\textbf{主要讨论文献}} \\ 
\midrule
收入差异 &  &  \cite{kennanEffectExpectedIncome2011}得出收入差距是造成劳动力流动的最显著因素\\
房价 &  &  \cite{ZhangLiFangJieRuHeYingXiangLaoDongLiLiuDong2017}\\
户籍 &  & \cite{ngaiChinasMobilityBarriers2019}讨论户籍的影响 \\
公共服务 &  教育资源&  \cite{XiaYiRanChengShiJianDeMengMuSanQianGongGongFuWuYingXiangLaoDongLiLiuXiangDeJingYanYanJiu2015}讨论城市公共服务对于移民的影响\\
  & 医疗资源 & \\
  & 生活资源 & \\
  & 商业丰富度 & \\
气候 &  &  \cite{HongDaYongDiWeiChaiYiGuaYingXingYuJiXiaoQiDaiKongQiWuRanYouZhiDeJuMinQianChuYiXiangFenYiYanJiu2016}讨论空气质量对于居民迁出意愿的影响\\
自然灾害 &  &  \\
迁移成本 &  &  \cite{WangLiLiWoGuoRenKouQianYiChengBenChengShiGuiMoYuShengChanLu2020}讨论了迁移成本的影响\\
文化壁垒 & 方言 &  \cite{LiuYuYunLaoDongLiKuaFangYanLiuDongDeDaoUXingMoShi2015}研究方言的U型影响\\
  & 生活习惯 &  \\
  & 社会认同 &  \\
地区偏好 &  &  \\ 
年龄 & & \\
\bottomrule
\end{tabularx}
\caption{讨论影响劳动力迁移因素的主要文献}
\label{tab:影响劳动力迁移的因素}
\end{table}



And with so many discussions at depth and length, a concept called migration friction has emerged from researches unanimously. 


\textbf{迁移摩擦(劳动力壁垒)}

\textbf{(整理大部分对于迁移摩擦的认识,同时在文末,根据实证结果对于哪些因素是迁移摩擦给出了统一的结论。)}

对于劳动力的迁移摩擦,目前的学术界还缺乏统一的定义。

\cite{leeTheoryMigration1966}在推拉理论的基础上加入中间障碍因素和个人因素。同时,他认为流入地和流出  地各自都有推和拉两种力量。中间障碍因素主要包括制度安排、距离远近、文化差异等,在中国尤其表现为户籍制度的影响。

近年来,大量学者从经典的推拉理论出发,基于推力、拉力与成本等角度讨论了中国跨地区劳动力流动的影响因素(邢春冰等,2013;彭国华,2015;张莉等,2017)。在转型经济背景下,成本因素对中国跨地区劳动力流动的影响逐渐成为现有研究的重点内容。越来越多的研究者从户籍制度、房价、基础设施与方言壁垒等因素出发,基于个体角度探讨了影响跨地区劳动力流动的决定机制(马伟等,2012;梁琦等,2013;刘毓芸等,2015),为我们认识和理解中国跨地区劳动力流动提供了重要的理论基础和文献支撑。但这些研究主要从某个具体影响因素出发,对于各个决定因素在跨地区劳动力流动中的重要性评估仍存在不足。在这些文献的基础上,本文从自然壁垒与制度壁垒角度出发,对跨地区劳动力流动壁垒进行分解,从而对劳动力流动壁垒的决定因素进行系统分析,进一步推动了相关研究在这一领域的进展。

\cite{JiangWeiZhongGuoKuaDiQuLaoDongLiLiuDongBiLeiCeDuFangFaYanJinQuShiYuJueDingYinSu2024}指出\textit{长期以来,户籍、地域、身份、档案、人事关系等因素构成了跨地区劳动力流动的重要障碍,阻碍了劳动力在地区间的有效循环和优化配置,造成了地区间严重的资源空间错配(梁琦等,2013;潘士远等,2018;黄文彬和王曦,2020;陆铭和李鹏飞,2022)。}同时\textit{将跨地区劳动力流动壁垒界定为能够阻碍或限制个体劳动力在不同地理区域之间自由流动的各类流动壁垒总和,其中包括了交通和通信成本、住房成本、政策福利等自然壁垒和制度壁垒。一方面,地理距离、文化多样性、地区信任等自然壁垒将提升跨地区劳动力流动成本,从而阻碍跨地区劳动力资源配置;另一方面,土地、户籍、身份、档案、人事关系等构成的制度壁垒也是影响劳动力跨地区流动的关键因素。由于跨地区劳动力流动壁垒具有不可观测以及微观不可加总的特征,劳动力流动壁垒的测度始终是现有研究中的一个难点问题。}

\textit{近年来,大量研究基于推拉理论讨论了中国劳动力跨地区乃至跨国流动的决定因素,为理解劳动力流动壁垒的形成机制提供了重要的理论支撑(马伟等,2012;都阳等,2014;Ngaietal.,2019),这些研究主要从个体微观层面对劳动力流动壁垒的影响因素进行分析。}

大量文献从不同的微观机制出发,研究分析了一系列影响跨地区劳动力流动壁垒的因素。其一,从自然因素角度出发,国内外学者分别探索了交通成本、文化成本以及心理成本对于跨地区劳动力流动壁垒的影响(Hayashi\&Prescott,2008;Beegleetal.,2011;马伟等,2012;Munshi\&Rosenzweig,2016)。其二,从政策因素角度出发,一些学者针对中国的户籍制度、社会保障、人事关系等相关制度,对跨地区劳动力流动壁垒的决定机制展开深入讨论(Cao\&Birchenall,2013;都阳等,2014;Ngaietal.,2019;Adamopoulosetal.,2022)。但这些研究大都是基于某个具体的微观因素,讨论劳动力流动壁垒对微观个体跨地区流动的影响,难以评估各个决定因素在跨地区劳动力流动中的重要性,而且无法对各类因素的影响进行加总,导致从宏观层面理解跨地区劳动力流动壁垒的研究仍然相对匮乏。

负面效果的推力可以被视作迁移摩擦

\cite{WangLiLiWoGuoRenKouQianYiChengBenChengShiGuiMoYuShengChanLu2020}讨论了迁移摩擦这一概念,可以将迁移成本认为是(引用)\textit{进一步降低人口流动壁垒将有利于我国城市规模的扩张与劳动力资源配置效率的改进。}

虽然在人口红利逐渐衰减的驱动下,地方政府纷纷加大了对人才的抢夺力度,出台了一系列人才吸引政策,但由于人才政策也只能覆盖一小部分劳动力群体,劳动力在地区间流动无论在流入地还是流出地仍然面临着较高的壁垒。

\cite{HanQiHengNongCunLaoDongLiQianYiMoCaYingXiangNongMinGongShuLiangYuGongZiJieGouMa2018}:
\textit{横亘在中国城乡迁移之间的迁移摩擦究竟是什么?其究竟是如何影响中国迁移人员的动态规律的,如何影响迁移人员的数量结构、教育结构和工资结构? 迁移摩擦是否可以揭示中国为什么会出现“迁移之谜”现象?}
韩认为劳动力迁移的摩擦由迁移成 本、城镇失业工资( 保险金) 和暂时性迁移工人工作搜寻匹配摩擦组成。

\cite{LiuXiuYanFangJieQianYiMoCaYuZhongGuoChengShiDeGuiMoFenBuLiLunMoXingYuJieGouShiGuJi2017}
\textit{消除城市间的房价差异几乎不影响人口的再配置,而消除迁移摩擦则会导致大规模的人口重新配置和带来显著的福利增进效应,这意味着迁移摩擦的存在是造成中国城市体系扁平化的关键致因。因此,全面推进户籍制度改革,有序放开城市的落户限制,进一步降低人口迁移中的空间摩擦,才能有效发挥市场的内生化力量,促进城市体系空间布局的优化。}


程名望2007






































% ---------------------------------------- 理论模型 ----------------------------------------
\chapter{理论模型}

劳动力流动问题本质上极为复杂。经过前述分析,我们认识到,劳动力流动和迁移不应被简单地视为一次性经济活动,而应被理解为贯穿个体整个生命周期的持续决策过程。个体有在多个地理位置之间自主选择的自由意志,每个位置均对应一种独特且互斥的收益流(payoff flow)。当个体\footnote{本文中出现的个体指可以独立做出决策的经济单位,其具体形式可能为个人,也可能为家庭。为了消除歧义,后文将尽可能使用“决策者”一词代替。}做出迁移决策后将承担相应的迁移成本。决策者在不同阶段根据自身状况和外部环境变化,不断权衡是否迁移、迁移到何处,以及迁移后如何实现自身利益最大化。

本文通过动态离散选择框架构建了迁移序列动态决策模型,其建模必要性源于三重内生性挑战:首先,传统静态模型无法捕捉生命周期中人力资本积累引致的迁移倾向时变特征;其次,观测数据中存在不可观测的个体异质性(如风险偏好、信息处理能力)与迁移决策互为因果;第三,区位间收益流具有时变协整性,需构建跨期关联约束下的预期效用函数。为此,本研究提出动态迁移决策模型(Dynamic Migration Decision Model, DMDM)。

\section{初步模型}

考虑一个封闭经济体,
人口流动仅发生在经济体内部,不存在国际移民或与外部世界的人口交换。
人口在各地区之间分布不均,且随时间动态变化。
假设人口的总量保持不变。
经济体由 $m$ 个离散地区组成,形成迁移网络$\mathcal{C} = \{1,2,\dots,m\}$,即个体决策者的居住地选择集。
各区域拥有固定的地理边界和空间关系,每个地区连通且可达,迁移连通性由$m$行$m$列的邻接矩阵$Adj=[j_{xy}\in\{0,1\}]$刻画。
不同的区域都具有独特的地区特征,这包括经济特征与非经济特征等,每个地区的特征会影响个体的迁移决策。
各地区的工资由本地劳动力市场决定。
时间被划分为离散的时期$t=\{0,1,2,\ldots,t,t+1,\ldots\}$。决策者获取效用,在每个时期以效用最大化为目标做出迁移决策。地区特征(如工资、就业机会)、经济状况和政策随时间变化,影响个体的预期和决策。

同时经济体中实施户籍制度。
令居民的出生地为其户口所在地$hukou$。
如果决策者的户口所在地与其当前所在不同,就会对其福利、基础设施的获取造成负面影响,从而降低效用的获取。反之,则会增加效用。户籍制度对于决策者的影响效果如图\ref{fig:户籍制度造成影响的途径}所示。
\footnote{
这样的经济体会产生迁移吗?
在劳动力自由流动的经济体中,地区间收入差异会因劳动力迁移而趋于收敛,但非经济因素的地区差异仍然存在。即使在非经济因素趋同的特殊情形下,由于决策者的地区偏好设定,劳动力流动仍将持续存在。此外,本文还引入了户籍制度障碍、家乡溢价和迁移成本,进一步制约了劳动力的自由流动,降低了地区收入差异收敛的能力。
}

\begin{figure}[!ht]
\centering
\includegraphics[width=0.8\textwidth]{images/户口影响.drawio.pdf}
\caption{户籍制度造成影响的途径}
\label{fig:户籍制度造成影响的途径}
\end{figure}


假设资产不影响居民的选择,考虑一个理性的决策者在$T$期生命周期内选择最优居住地序列。令$j_t \in \mathcal{C}$表示第$t$期的居住地选择,那么完整的选择序列可表示为$\mathcal{J}=\{j_1, j_2 ,\dots,j_T\}$。典型的居住地选择模式如表格\ref{tab:居住地选择序列可能的形式}所示。

\begin{table}[!ht]
\centering
\begin{tabular}{@{}ll@{}}
\hline
\multicolumn{1}{c}{\textbf{个体流动形式}} & \multicolumn{1}{c}{\textbf{居住地选择序列}}    \\ \midrule
\begin{tabular}[c]{@{}l@{}}\textbf{长期定居}:决策者在特定时间段内总是\\ 停留在同一个地区\end{tabular}                          & $\mathcal{J}=\{\dots, 1,1,1,1,1,\dots\}$ \\
\begin{tabular}[c]{@{}l@{}}\textbf{单向迁移}:如果 $j_p \neq j_q, \forall p\neq q$,那么\\ 个体就进行了迁移\end{tabular} & $\mathcal{J}=\{\dots, 1,2,3,4,5,\dots\}$ \\ 
\textbf{回流迁移}:返回到先前居住过的地区                      & $\mathcal{J}=\{\dots, 1,2,2,1,1,\dots\}$ \\ \bottomrule
\end{tabular}
\caption{居住地选择序列可能的形式}
\label{tab:居住地选择序列可能的形式}
\end{table}


决策问题由状态变量$x_t$刻画,其包含
个体特征(年龄、收入、家庭规模等)、
地区属性(房价、公共设施等)、
历史居住选择记录、
宏观经济环境
等因素。
状态转移服从马尔可夫过程,其转移概率为$p(x_{t+1}|x_t,j_t)$。

决策者在第$t$期从居住地$j$
\footnote{本文中$j$代表可选择的地点,而$j_t$代表个体在$t$期做出的具体选择。}
获得的地区效用为:
\begin{equation}
  \tilde{u}_t(j, x_t) = u_t(j, x_t)  +  \zeta_{jt}
  \label{eq:地区效用函数}
\end{equation}
其中$u_t(j, x_t)$为确定性效用项,
$\zeta_{jt}$为在任意时期和地点都独立同分布的随机扰动项,服从标准Gumbel分布$F(\zeta) = \exp(-\exp(-\zeta))$,且与状态变量$x_t$无关。


理性决策者做出选择的标准是选择使得效用最大化的选项:
\begin{equation}
  j_{it} = \arg\max_{j \in C} \{\tilde{u}_t(j, x_t)\}
\end{equation}

对于在选择集$\mathcal{C}$中选择$j$的概率可以写为:
\begin{equation}
\begin{split}
  Pr(\text{选择地点j})&=Pr(\tilde u_j > \tilde u_k, \forall k \neq j)
  \\&=Pr(u_j+\zeta_j>u_k+\zeta_k, \forall k \neq j)
  \\&=Pr(\zeta_k-\zeta_j<u_j-u_k, \forall k \neq j)
\end{split}
\label{eq:C中地点选择j的概率}
\end{equation}

在动态框架下,决策者具有前瞻性,并且通过最大化期望折现效用进行跨期优化。决策者在每期进行最优的居住地选择,或者说决策者选择$T$维的最优居住地选择序列$\mathcal{J}^*=\{j_1^*,j_2^*,\ldots,j_T^*\}$。每期的效用可以分为两部分,一是选择地点$j \in \mathcal{C}$带来的当期效用,二是折现的通过转移概率加权得到的未来期望效用。

假设折现因子为$\beta \in (0,1)$,决策者的目标是为从$t$期开始的期望折现效用最大值,这可以表示为:
\begin{equation}
  V_t(x_t, \zeta_{j_t}) = \max_{j_t \in \mathcal{C}} 
  \left\{ 
  \tilde{u}(j_t, x_t) + \beta \sum_{x_{t+1}} p(x_{t+1} | x_t, j_t) \cdot \mathbb{E}_{\zeta_{j_{t+1}}} [ V_{t+1}(x_{t+1}, \zeta_{j_{t+1}}) ]
  \right\}
\end{equation}
其中$\mathbb{E}_{\zeta_{j+1}} v_{t+1}(x_{t+1},\zeta_{j+1})$
为在状态 
$x_{t+1}$下,对所有可能随机扰动 
$\zeta$取期望后的长期效用。

那么决策者的基本问题就是在每个给定状态$x$和随机项 $\zeta_j$的情况下,选择最优居住序列$\mathcal{J}^*$使得全生命周期中的期望折现效用最大化,即:
\begin{equation}
  \max_{\mathcal{J}=\{j_1,j_2,\ldots,j_T\}} \mathbb{E} [ \sum_{t=1}^{T} \beta^{t-1} \tilde{u}_{j_t}(j_t,x_t) ]
\end{equation}

也可以使用迭代形式,即决策者每期都选择$j$来实现使得个体在生命周期中的总效用最大化,
令
$\begin{cases}
  v(j_{t},x_{t})=u(j_{t} ,x_{t})+\beta \sum_{x_{t+1}} p(x_{t+1}|j_t,x_t) \cdot \mathbb{E}_{\zeta_{j+1}} V_{t+1}(x_{t+1},\zeta_{j+1})
  \\
  \bar v(x_{t})=\mathbb{E}_{\zeta_{j_t}} v_{t}(x_{t},\zeta_{j})
\end{cases}$
,其中$E_{\zeta}$表示对 J 维向量$\zeta$的分布求期望\footnote{对于决策者而言收益冲击$\zeta$在做出居住地选择决策之前就已经实现,而$v(x,j)$ 和 $\bar v(x)$ 之间的区别在于,一个表示每个替代选择的延续值,而另一个表示在收益冲击实现之前所取的优化延续值的期望。},决策者基本问题的迭代形式为:
\begin{equation}
V(x,\zeta)=\max\limits_{j}[v(j,x)+\zeta_{j}]
\end{equation}

图片\ref{fig:migration_flow_resized2} 展示了一个动态规划决策过程。在离散周期$t$结束时或者下个周期$t+1$即将开始时,决策者根据在抽取到的随机收益冲击$\zeta$和状态转移概率$p(x_{t+1}|x_t,j_t)$双重不确定性的影响下通过Bellman方程做出决策,进入下一个周期$t+1$后获得该期的效用$\tilde u_{t+1}$。

\begin{figure}[!ht]
\centering
\includegraphics[width=0.9\textwidth]{images/dynamicsequence2.drawio.pdf}
\caption{动态离散选择模型下的劳动力迁移决策流程图}
\label{fig:migration_flow_resized2}
\end{figure}

由于随机效用项$\zeta_j$独立同分布且服从 I 型极值分布,根据\cite{mcfaddenConditionalLogitAnalysis1973,rustOptimalReplacementGMC1987,rustStructuralEstimationMarkov1994}的结论,可知:
\begin{equation}
  \exp(\bar v(x_t))=\sum\limits_{k=1}^{J}\exp(v(x_t,k_t))
\end{equation}
可知选择地点j的概率,即公式\ref{eq:C中地点选择j的概率},存在softmax函数形式,这是由Gumbel噪声的极值分布性质推导的:
\begin{align}
\rho(x,j)&=\exp(v(x,j)-\bar v(x))
\\&=\frac{\exp(v(x,j))}{\sum\limits_{k\in J} \exp(v(x,k))} \label{eq:地点选择概率}
\end{align}
这极大地优化了动态规划问题中期望值的计算,并且由于sotfmax函数输出的概率分布是平滑的,有利于策略迭代的稳定性。


%由于随机效用项$\zeta_j$独立同分布且服从 I 型极值分布,其概率累积分布为$F(\zeta_j)=\exp(-\exp(-\zeta_j+\bar \gamma))$,其中$\bar \gamma$为欧拉常数。

%$\max_j \left(v(x,j)+\zeta_j\right)$的期望为$\bar v(x) = E_\zeta V(x,\zeta)= \ln(\sum\limits_{k=1}^J \exp(v(x,k)))+\bar \gamma$。这表明$\exp(\bar v(x)-\bar \gamma)=\sum\limits_{k=1}^J \exp(v(x,k))$。将其简化为,$\exp(\bar v(x))=\sum\limits_{k=1}^J \exp(v(x,k))$。

%利用Gumbel分布的封闭性质,将选择地点$j$的概率事件表达式化为$\rho(x,j)=\exp(\bar \gamma + v(x,j) - \bar v(x))$,
%将$\bar v(x)$的定义式代入其中,化简后利用指数运算的特性可得动态离散选择中的地点选择概率\footnote{这实际上也是softmax函数的一种概率向量形式,可以在动态模型中通过$\bar v(x) = \ln (\sum \exp(v(x,k)))+\bar \gamma $将未来效用递归嵌入当前选择。}
%\begin{equation}
%  \rho(x,j)=\frac{\exp(v(x,j))}{\sum\limits_{k=1}^J \exp(v(x,k))}
%\end{equation}


\section{效用函数具体构成}

接下来为公式\ref{eq:地区效用函数}赋予具体形式,从而使其具有经济意义。
假设个体决策者$i \in \mathcal{N}$的
家乡为$h$,那么该点上可能的效用流为:
\begin{equation}
  \tilde u_{t}^h(j_t,x_t;\theta)=u_{t}^h(j_t,x_t;\theta) +\zeta_{j_t,t}
  \label{eq:家乡效用函数}
\end{equation}

假设收入的边际效用是固定的,居民可以在固定利率下自由借贷,那么居民的期望效用最大化问题就可以转化为预期永久收入\footnote{永久收入以现值形式记录。}
减去一次性付清的迁移成本。
\footnote{
假设效用函数的线性形式为$U(x)=a X$,其中a为边际效用参数。居民可在固定利率$r$下无限制借贷,预算约束满足消费现值等于收入现值。
居民可以留在原地(永久收入为$Y_h$)或迁移(永久收入为$Y_p$,后者需支付一次性迁移成本 $M$。
假设迁移成本 
$M$
为即期支出,且永久收入为永续年金。

若居民留在原地,其永久收入的现值为
$W_h = \sum\limits_{t=0}^\infty \frac{Y_h}{(1+r)^t}=\frac{Y_h}{r}$,
总效用为$U_h=a W_h = \frac{a Y_h}{r}$。
若居民选择迁移,需支付即期成本 
$M$
,迁移后的永久收入现值为
$W_h = \sum\limits_{t=0}^\infty \frac{Y_p}{(1+r)^t}=\frac{Y_p}{r}$,
由于迁移成本 
$M$
为即期支出,净现值为
$\mathcal{W}=W_p-M=\frac{Y_p}{r}-M$,
总效用为
$U_p=a(\frac{Y_p}{r}-M)$
迁移的条件为$U_p>U_h \Rightarrow a(\frac{Y_p}{r}-M) > a \frac{ Y_h}{r} \Rightarrow Y_p-Y_h > rM$。
期望效用最大化等价于选择净现值更高的选项,即$\max\{a(\frac{Y_p}{r}-M), a \frac{ Y_h}{r}\}$。
}
居民的基础效用由货币性收入、城市宜居度、家乡溢价减去迁移成本组成。令效用函数为如下形式:
\begin{equation}
  \begin{split}
    u_t^h(j_t,x_t;\theta)=&\alpha_0 \cdot w_{ijt}+\sum\limits_{s} \alpha_{s} \cdot y_{s}(j_t)  + \alpha_h \cdot I(j_t=h) 
    \\& + \alpha^p \cdot I(j_t \neq hukou) +\xi(j_t,\omega)-\kappa(j_t,j_{t-1},x_t)
  \end{split}
  \label{eq:家乡效用函数中的具体构成}
\end{equation}
其中$\theta$代表参数向量, $w(j_t,\omega_t)$为经济收益函数,衡量个体在当前位置通过劳动获得的经济收益,$\alpha_0$表示工资收入对效用的权重(边际效用);
$y_{s}(j_t)$为该居住地提供的非货币性舒适度(amenity)价值,如气候、公共服务、文化等,反映地理位置的非经济吸引力,$y_s$是第$s$种舒适度的量化指标(如教育质量指数),其边际效用由$\alpha_s$表示;
$\alpha_h \cdot I(j_t=h)$是对家乡的偏好溢价,$\alpha_h$衡量个体对出生地的情感依恋强度,其中$I(\cdot)$为指示性函数,当其中判断为真时返回1,否则为0,这意味着只有当当前所在地是家乡时效用才会额外增加$\alpha_h$;
$\alpha^p \cdot I(j_t = hukou)$为人户分离时的制度障碍效应,只有当所在地与户籍地不同时才会产生$\alpha^p$单位的效用增减;
$\xi(j_t,\omega)$为个体的地区偏好成分,
反映个体对某地点的长期偏好匹配(如文化契合度),仅当个体访问该地点时才能获知其具体值;
$-\kappa$为从其他区域前往地区$j$的迁移成本函数,负号表示迁移成本会降低效用,包括直接费用(如搬家费)和间接成本(如适应新环境的心理成本)。



公式\ref{eq:家乡效用函数中的具体构成}中,决策者的经济收益函数由多种因素组成。在不考虑时间因素的影响下,个体决策者$i \in \mathcal{N}$在地区$j \in \mathcal{C}$的收入设定为以下函数形式:
\begin{equation}
  w_{ij}=\mu_j + \nu_{ij} + G(X_i,a,t) + \eta_i + \varepsilon_{ij}(a)
  \label{eq:经济收益函数}
\end{equation}
地区基准收入$\mu_j$反映地区$j$的平均工资水平,
代表地区的经济基本面,如生活成本、区域产业聚集效应(如北京的传媒业、山西的采矿业)、政策红利(税收减免、人才补贴)等结构性特征,是所有在该地点个体的工资基准,
其与个体无关,仅取决于地点特征。
个体-地区匹配效应$\nu_{ij}$表征相同技能劳动者在不同地区的收入差异,
捕捉个体与地区的特殊适配性(如技能需求匹配、文化契合度),例如,程序员在科技中心可能获得更高的$\nu_{ij}$,
其具备匹配效应在个体驻留期间保持稳定,一旦个体进入地区 $j$,
$\nu_{ij}$固定不变,但迁移到新地区$k$ 时会生成新的 $\nu_{ik}$\footnote{在这点上与地区偏好$\xi$类似,但前者可通过工资和迁移选择共同推断,而后者仅能通过迁移选择推断。}。
匹配效应实现值$\nu_{ij} \sim F_\nu$在个体进入地区$j$后方可观测。
生命周期收入$G(X_i,a,t)$为线性函数,其包含三重要素,
个体特征$X_i$包含教育程度、性别等,
年龄效应$a$反映人力资本积累曲线,
时间趋势$t$捕捉技术进步等外生冲击。
量化系统性因素对工资的影响(如工作经验随年龄增长、宏观经济波动)。
$G$ 的影响在不同地区间相同(如教育回报无地区差异)。
个体固定效应$\eta_i$表征表示个体 $i$ 异质性中不随地区改变的固有能力或特质,
如先天能力、家庭背景等,无论迁移到哪里均保持恒定。
假设个体已知 $\eta_i$ 的值,并在决策中利用此信息。
暂时性随机效应$\varepsilon_{ij}(a)$代表短期波动或随机扰动项,服从i.i.d分布的当期扰动,
捕捉不可预测的工资波动(如临时绩效波动、经济冲击)。
假设$\varepsilon$服从期望为0的正态分布,并且独立于其他变量,随年龄和时间变化,个体无法预知其实现值\footnote{所以即使暂态效应对于每个地点都不同,但对居住地选择决策无持续影响。}。

如果发生了迁移($j_t \neq j_{t-1}$),同个决策者$i$的迁移成本$\kappa$设定为以下函数形式:
\begin{equation}
\begin{split}
  \kappa=I(j_t \neq j_{t-1}) \cdot [& 
  \gamma_{0 \tau}
  + \gamma_1 D(j_{t-1},j_t)
  - \gamma_2 I(j_t \in Adj(j_{t-1}))
  \\&
  - \gamma_3 I(j_t \in Pre)
  + \gamma_4 a
  -\gamma_5 n_j
  ]
\end{split}
\label{eq:迁移成本函数}
\end{equation}
$\gamma_{0\tau}$迁移中类型相关的的固定成本,随个体类型 $\tau$ 变化,从而捕捉未观测的异质性。例如:  
“定居型”个体(stayer type)的 $\gamma_{0\tau}$ 极高,几乎禁止迁移;
“流动型”个体的 $\gamma_{0\tau}$ 较低,更易迁移。不同类型的群体分布由对应的概率$\pi_\tau$给出,并且满足$\sum\limits_{\tau}^{} \pi_\tau=1$。
$\gamma_1 D(\ell^0, j)$为迁移距离效应,是 $D(\ell^0, j)$ 的线性成本(以大圆距离衡量),距离越远,成本越高($\gamma_1 > 0$)。  
$-\gamma_2 \chi(j \in A(\ell^0))$是相邻地区效应,若目标地点 $j$ 与当前地点 $\ell^0$ 相邻(即 $j \in A(\ell^0)$),则成本减少 $\gamma_2$。
相邻地区迁移可能因文化相似、信息透明或交通便利而成本更低($\gamma_2 > 0$)。  
$-\gamma_3 \chi(j = \ell^{-1})$为历史地区效应(回流效应),
其定义为若目标地点 $j$ 是个体曾居住过的历史地点 $\ell^{-1}$,则成本减少 $\gamma_3$。
这是因为返回熟悉地区可能因现存社会网络或无需重新适应而成本更低($\gamma_3 > 0$)。  
$+\gamma_4 a$代表年龄障碍效应,表明迁移成本随年龄 $a$ 线性增加,年龄越大,迁移的心理或生理成本越高($\gamma_4 > 0$)。  
$-\gamma_5 n_j$为目标地区规模成本效应,
目标地区人口规模 $n_j$ 越大,迁移成本减少 $\gamma_5 n_j$。  
这是因为
大规模地区可能有更多潜在支持(如亲友网络、服务机构),减少安家难度,从而降低迁移成本($\gamma_5 > 0$),符合“引力模型”中人口规模吸引迁移的经典结论。  
$I(j_t \neq j_{t-1})$为迁移指示函数 ,仅当目标地点 $j \neq \ell^0$ 时,迁移成本生效(否则为0)。  这确保个体留在原地时无需支付迁移成本。

迁移成本中参数的经济含义如表格\ref{tab:迁移成本参数释义}所展示。
\begin{table}[!ht]
  \centering
  \begin{tabularx}{\textwidth}{@{}llX@{}}
    \toprule
    \multicolumn{1}{c}{\textbf{参数符号}} & \multicolumn{1}{c}{\textbf{参数含义}} & \multicolumn{1}{c}{\textbf{参数意义}} \\ \midrule
    \multicolumn{1}{c}{$\gamma_{0\tau}$} & 类型异质效应 & 捕捉不同类型的异质性,可以解释为何有些人从不迁移,识别“定居型”人群可针对性地设计激励措施(如搬迁补贴)。\\ 
    \multicolumn{1}{c}{$\gamma_1$} & 物理迁移成本 & 个体与其携带物必然因为物理法则而产生运输成本。 \\ 
    \multicolumn{1}{c}{$\gamma_2$} & 相邻地区效应 & 反映地理邻近性的普遍优势。 \\ 
    \multicolumn{1}{c}{$\gamma_3$} & 历史地区效应 & 返回历史地区因熟悉度获得折扣,体现路径依赖。 \\ 
    \multicolumn{1}{c}{$\gamma_4$} & 年龄障碍效应 & 针对不同年龄群体制定差异化迁移政策(如青年人才引进计划)。\\ 
    \multicolumn{1}{c}{$\gamma_5$} & 规模成本效应 & 大地区吸引力不仅体现在经济机会(如工资 $\mu_j$),也直接降低迁移成本。\\ \bottomrule
  \end{tabularx}
  \caption{迁移成本参数释义}
  \label{tab:迁移成本参数释义}
\end{table}


此外,由于暂态效应$\varepsilon$服从期望为0的正态分布,已知经济收益函数设定为公式\ref{eq:经济收益函数}。
令$\phi$表示标准正态分布的概率密度函数,处于地点$j$的个体$i$在时期$t$的经济收益函数的密度函数$\Psi$为:
\begin{equation}
  \Psi_{it}=\phi(\frac{w_{it} - \mu_{jt} - G_{it} - \nu_{ijt} - \eta_i }{\sigma_{\varepsilon_{it}}})
  \label{eq:经济收益似然贡献}
\end{equation}


\section{状态转移概率}

状态变量转移概率描述的是在选择了特定位置$j_{t}$的情况下,从状态 $x_t$ 移动到新状态 $x_{t+1}$的条件可能性。
假设状态变量转移概率具有马尔可夫性,即系统的未来状态仅依赖于当前状态和当前决策,与过去的历史无关,这意味着:
\begin{equation}
  p(x_{t+1}|x_{t},j_{t},x_{t-1},j_{t-1},\ldots)=p(x_{t+1}|x_{t},j_{t})
\end{equation}
该假设,符合沉没成本理论,大大简化了问题的复杂度。

具体而言,本文将其设置为
\begin{equation}
  p(x_{t+1}|x_t,j_t)=
  \begin{cases}
    % 迁移
    \frac{1}{m^\omega}
    , &\text{发生迁移从而产生随机抽取}
    \\
    % 不迁移
    1
    , &\text{没有迁移}
    \\
    % 其他
    0
    , &\text{其他情况}
  \end{cases}
\end{equation}

分情况解析:
情况1:不迁移($j = l^0$)
转移规则:新状态 $\tilde{x}'$ 与原状态 $\tilde{x}$ 完全相同。年龄 $a' = a + 1$。
个体留在当前地点,所有匹配项(工资 $v_{il^0}$ 和效用 $\xi_{l^0}$)保持不变。
年龄增长反映时间推移,但状态向量其他部分无变化。

情况2:迁移到前一个地点($j = l^1$)
转移规则:
新地点 $l'^0 = l^1$,原当前地点变为前一个地点 $l'^1 = l^0$。
工资和效用匹配项的索引交换:  
$x_0'^0 = x_0^1$(原前一个地点的工资匹配项变为当前地点的匹配项),  
$x_0'^1 = x_0^0$(原当前地点的工资匹配项变为前一个地点的匹配项),  
同理,效用匹配项索引 $x_\xi'^0 = x_\xi^1$,$x_\xi'^1 = x_\xi^0$。
年龄 $a' = a + 1$。
个体返回历史地点时,状态变量中的“当前”与“历史”位置互换。
保留原匹配项的索引,避免重复抽取,体现路径依赖。

情况3:迁移到新地点($j \notin \{l^0, l^1\}$)
转移规则:
新地点 $l'^0 = j$,原当前地点变为前一个地点 $l'^1 = l^0$。
工资匹配项索引 $x_0'^0 = s_0$(从 $1$ 到 $n_0$ 随机抽取),原前一个地点的工资匹配项索引保留 $x_0'^1 = x_0^0$。
效用匹配项索引 $x_\xi'^0 = s_\xi$(从 $1$ 到 $n_\xi$ 随机抽取),原前一个地点的效用匹配项索引保留 $x_\xi'^1 = x_\xi^0$。
年龄 $a' = a + 1$。
发生概率为 $\frac{1}{n^2}$,假设 $n_0 = n_\xi = n$。
迁移至全新地点时,需重新抽取该地点的工资和效用匹配项(体现不完全信息)。
保留原当前地点的匹配项作为历史记录(如 $x_0'^1 = x_0^0$),确保未来可返回时复用。

情况4:其他情况(概率为0)
若迁移目标不符合上述条件(如无效地点),状态转移概率为0。




状态变量$x$囊括了M维最近地区向量$\ell$,$m$维地区工资与地区效用信息$\omega$和年龄$a$。出于简便计算,规定其状态变量转移概率为:
\begin{equation}
p(x_{t+1}|x_t, j_t) = 
\begin{cases} 
1, & \text{if } j_{t+1} = j_t, \tilde{x}' = \tilde{x}, a' = a + 1 \\
1, & \text{if } j_{t+1} = j_{t-1}, \tilde{x}' = (l^1, l^0, x_0^1, x_0^0, x_\xi^1, x_\xi^0), a' = a + 1 \\
\frac{1}{n^2}, & \text{if } j_{t+1} \notin \{j_t, j_{t-1}\}, \tilde{x}' = (j, l^1, s_0, x_0^0, s_\xi, x_\xi^0), \quad 1 \leq s_0 \leq n_0, \quad 1 \leq s_\xi \leq n_\xi, \quad a' = a + 1 \\
0, & \text{otherwise}
\end{cases}
\end{equation}

状态向量定义:
\[ x = (\tilde{x}, a) \quad \text{其中} \quad \tilde{x} = (l^0, l^1, x_0^0, x_0^1, x_\xi^0, x_\xi^1) \]
- 变量解析:
  1. $j_t$:当前所在地点。
  2. $j_{t-1}$:前一个地点(迁移前的历史地点)。
  3. $x_0^0$:当前地点 $l^0$ 的工资匹配项($v_{il^0}$)的索引(即其具体实现值的标识符)。
  4. $x_0^1$:前一个地点 $l^1$ 的工资匹配项的索引。
  5. $x_\xi^0$:当前地点 $l^0$ 的效用匹配项($\xi_{l^0}$)的索引。
  6. $x_\xi^1$:前一个地点 $l^1$ 的效用匹配项的索引。
  7. $a$:个体年龄。

参数与符号说明:
- $n_0$ 和 $n_\xi$:工资匹配项 $v$ 和效用匹配项 $\xi$ 的可能实现值数量(假设为离散分布)。
- $s_0$ 和 $s_\xi$:从离散分布中随机抽取的索引值,分别对应新地点的工资和效用匹配项。
- $\frac{1}{n^2}$:迁移到新地点的概率,假设所有新地点的匹配项独立且均匀分布。

信息更新:个体仅在实际迁移到某地点后,才能获知其匹配项的真实值(通过索引 $s_0, s_\xi$ 的抽取)。
状态压缩:通过索引(而非具体值)记录匹配项,简化状态空间,便于数值求解。
动态决策:迁移决策基于当前和历史信息,但未来新地点的匹配项需通过随机探索学习。







\section{劳动力迁移决策}

由于地区偏好效应和个体固定效应仅与个体特征相关,而暂态效应仅在迁移至目的地后随机产生,因此这两个因素不会影响决策者的迁移选择。
在不同地区$j$和$k$之间,仅有平均工资$\mu$、地区匹配效应$\nu$和非经济因素存在地区间差异。
地区选择通过影响收入水平而产生作用,劳动力将倾向于选择具有更优劳动力市场条件、更高地区匹配效应和更高宜居度条件的目的地。
然而,这种选择必须同时满足预期收益需超过户籍制度障碍、迁移成本、心理成本(如恋家效应)以及随机效用冲击等因素造成的阻力。
简言之,外部条件对决策者效用的提升必须足以克服迁移过程中的各类障碍,这恰如公式\ref{eq:迁移motivation}所示。
\begin{equation}
\begin{split}
  Pr(\text{选择k})&= Pr(u_k - u_j > \zeta_j - \zeta_k) 
  \\&= Pr(
  \text{地区k的客观因素} - \text{地区j的客观因素} 
  > 
  \underbrace{\alpha^p}_{\text{制度}} + \underbrace{\alpha^h}_{\text{心理}} + \underbrace{\kappa}_{\text{迁移成本}} + \underbrace{
  \zeta_j - \zeta_k}_{\text{外生冲击}} 
  )
  \\&= p >0
\end{split}
\label{eq:迁移motivation}
\end{equation}

至此,模型刻画出了在有限生命周期内,每期决策者可以在迁移网络$\mathcal{C}$中自由选择最优住址$j_t$,这些不同时期的选择构建出了最优居住地选择序列$\mathcal{J}$。这意味着迁移可以是多次的,可撤回的。




% ---------------------------------------- 实证部分 ----------------------------------------
\chapter{实证处理}

\section{模型上} 

采取动态规划方法的一个缺陷在于计算上很容易陷入维度的诅咒(curse of dimensionality),出于实证处理的简便处理,本文将对于模型中一些难以计算的点进行简便假设。

\subsection{年龄效应}
出于简化
参照以往的计量经济学经验
将年龄效应设置为年龄的有待估参数的一次项和二次项的和
\ref{eq:家乡效用函数中的具体构成}中的$G$设置为
$$G=r_1 a + r_2 a^2$$


\subsection{非参数离散混合方法} 

基于上述模型设定,公式\ref{eq:家乡效用函数中的具体构成}中的地区偏好$\xi$、公式\ref{eq:经济收益函数}中的个体固定效应$\eta$、地区匹配效应$\nu$以及暂态效应$\varepsilon$的具体函数形式仍未确定。若无法解决这一识别问题,将难以准确计算决策者的效用函数和期望值。因此,有必要对这些随机项的概率分布施加合理的假设和约束条件。

出于计算上的简便,
本文使用支撑点离散化方法。
对于任何不知道其分布的 $F$,我们可以找到最佳近似分布 $\hat F$ 来近似。对于连续分布的变量$F$,离散的 $\hat F$ 是一组有限的支撑点组成,每个支撑点都有对应的概率值,即$\{(x_i, p_i)\}, 1 \leqslant i \leqslant N$ 的组合,其中 $x_i$ 是离散点,$p_i$ 是其概率,$N$ 为给定的支持点数。
对于每个$q_r, (1\leqslant r \leqslant n )$,计算对应的效用值,用这些有限的效用值来代替连续效用的积分。
通过该处理后,模型变为有限状态的动态规划问题,可以直接构建似然函数并进行最大似然估计。
相较于蒙特卡洛模拟,该方法可以避免随机噪声,适合精确的极大似然估计。

由于假设分布围绕 0 对称,因此仅需要 $\frac{M-1}{2}$ 个支持点即可进行估计。

令 $p_i = \frac{1}{N}, \forall i$。


对于$\nu$
在每个地点中个体都有从地区匹配效应的分布中抽取一个值,假设这个分布是一个有限集合上的均匀分布,$Y=\{\nu(1),\nu(2)...\nu(n_{\nu})\}$
其结果是$\omega^{i}_{\nu}$,$\omega^{i}_{\nu}(j)$代表在地点j的匹配效应,$1\leqslant j\leqslant N_i$,$N_i$是$i$到达过的地方的数量
总共有$\{\omega^{i}_{\nu}(1),\omega^{i}_{\nu}(2)...\omega^{i}_{\nu}(j)...\omega^{i}_{\nu}(N_i)\}$

$\xi$
地区匹配偏好也是如此
来自于$\Xi=\{\xi(1),\xi(2)...\xi(n_{\xi})\}$
其结果是$\omega^{i}_{\xi}$

$\eta$ 表示每个个体工资的固定成分(例如长期能力或个体特质)。
假设其分布是一个对称的离散分布,具有7个支持点。每个支持点有相等的权重。
因为是离散分布,只需要估计 3个参数:分布的中间点(中心趋势)和离散范围。
固定效应代表长期的工资差异来源,例如教育水平、经验等。
选择离散分布是为了简化复杂的连续分布,同时通过支持点捕捉关键的分布特性(如中心趋势和差异性)。
$\eta$的离散分布提供了一种对长期工资特质的简化建模方式,减少计算复杂度,同时保留重要信息。
固定效应也是如此
来自于$H=\{\eta(1),\eta(2)...\eta(n_\eta)\}$
其结果是$\omega^{i}_{\eta}$


暂态效应$\varepsilon$在本质上是一种外生冲击,所以可以假设为服从期望为$0$的正态分布$N(0,\sigma^2)$,
保留了异方差特质以变容纳更多的异质性。
数学上$\varepsilon$表示工资的短期波动成分,比如由于经济周期、随机冲击等导致的变化
假设$\varepsilon_{it}$(某人在某个时间的短期波动)来源于一个均值为零的正态分布,其方差$\sigma_{\epsilon}$随人而异
对于每个人,$\sigma_\varepsilon(i)$来自一个离散分布(4 个支持点,均匀分布),因此需要估计 4 个参数
$\varepsilon$反映工资中不可预测的临时变化,例如经济不确定性或工作绩效波动
允许$\sigma_\varepsilon(i)$因人而异捕捉到工资短期波动的个体差异性(例如高风险行业可能有更大波动)
在建模中,$\varepsilon$允许方差随个体变化,体现了个体的异质性,这进一步增加模型的灵活性,使其能够更好地拟合数据。

暂态效应来自于一个期望为$0$的正态分布,但是假设其异方差使其可以可以容纳更多的异质性。
即$\varsigma=\{\sigma_{\epsilon}(1),\sigma_{\epsilon}(2)...\sigma_{\epsilon}(n_{\epsilon})\}$,
其结果是$\omega^{i}_{\epsilon}$。




表格\ref{tab:未观测到的变量表}总结了以上未知分布变量的设定。


\begin{table}[!ht]
\centering
\begin{tabular}{@{}ccccc@{}}
\toprule
变量 & 变量符号 & 分布假设 & 数量 & 抽取结果 \\ \midrule
地区匹配效应 & $\nu$ & $Y=\{\nu(1),\nu(2)\ldots\nu(n_{\nu})\}$ & n & $\omega^{i}_{\nu}(j)$\\
地区偏好 & $\xi$ & $\Xi=\{\xi(1),\xi(2)\ldots\xi(n_{\xi})\}$ & 1 & $\omega^{i}_{\xi}$ \\
固定效应 & $\eta$ & $H=\{\eta(1),\eta(2)\ldots\eta(n_\eta)\}$ & 1 & $\omega^{i}_{\eta}$ \\
暂态效应的方差 & $\varepsilon$ & $\varsigma=\{\sigma_{\varepsilon}(1),\sigma_{\varepsilon}(2)\ldots\sigma_{\varepsilon}(n_{\varepsilon})\}$ & 1 & $\omega^{i}_{\varepsilon}$ \\ \bottomrule
\end{tabular}
\caption{未观测到的变量表}
\label{tab:未观测到的变量表}
\end{table}


参数向量的可能实现组成一个集合,用$\Omega(N_{i})$表示,未观测到的因素组成一个$m_{i}+3$维的参数向量$\omega^{i}=(\omega^{i}_{\xi},\omega^{i}_{\eta},\omega^{i}_{\epsilon},\omega^{i}_{\nu}(1),\omega^{i}_{\nu}(2)...\omega^{i}_{\nu}(N_{i}))\in \Omega$。


对于公式\ref{eq:迁移成本函数}迁移成本函数中的异质性成本$\gamma_{0\tau}$,
模型假设随个体类型 $\tau$ 变化,从而捕捉未观测的异质性。
具体而言,本文将其分为分为3种类型。
类型1:高迁移倾向,低迁移成本,高不确定性容忍度。
类型2:中等迁移倾向,中等迁移成本,中等不确定性容忍度。
类型3:低迁移倾向,高迁移成本,低不确定性容忍度。
“定居型”个体(stayer type)的 $\gamma_{0\tau}$ 极高,几乎禁止迁移;
“流动型”个体的 $\gamma_{0\tau}$ 较低,更易迁移。
个体i属于类型k的概率为$\pi_k$,并且满足$\sum\limits_{k}^{} \pi_k = 1$。
每种类型的比例$\pi_1, \pi_2, \pi_3$作为模型参数被估计。
类型k的个体的效用函数为$u_{ijt}^k$,
模型可表示为:
$$V_{ijt}^k = u_{ijt}^k + \varepsilon_{ijt}^k + \beta^k E_t[V_{t+1}^k|d_{ijt}=1]$$
其中$k\in{1,2,3}$表示类型。


\section{似然函数}




由上文可知个体存在两种似然概率函数贡献,即公式\ref{eq:地点选择概率}和公式\ref{eq:经济收益似然贡献}。
通过个体在其历史轨迹中给出的两种似然贡献,
我们得到了类型为$\tau$的个体$i$的似然函数

即在所有可能的不可观测变量的组合下,个体在所有时期中的观测收入$w_{it}$上选择地点的概率$\rho_{it}$的乘积:
\begin{equation}
  L_{i}(\theta_{\tau})=\frac{1}{n_{\nu}n_{\epsilon}n_{\xi}(n_{\nu})^{N_{i}}} \sum\limits_{\omega^{i}\in\Omega(N_{i})}(\prod\limits_{i=1}^{T_{i}} \psi_{it}\lambda_{it})
\end{equation}

由于模型允许存在异质性

让$L_{i}(\theta_{\tau})$代表tau类型个体的似然函数,其中$\theta$是该个体的待估参数向量

样本的似然函数是一个混合类型的联合对数似然函数,把每个观测i的贡献相加
\begin{equation}
\Lambda(\theta)=\sum\limits_{i=1}^{N}\log(\sum\limits_{\tau=1}^{K}\pi_{\tau}L_{i}(\theta_{\tau})) 
\end{equation}
其中混合比例由$\pi_{\tau}$给出,且$\sum\limits_{\tau=1}^{K}\pi_{\tau}=1$
每个个体i都做出了贡献
这是在给定参数$\theta_{\tau}$的条件下,个体$i$在类型$\tau$下的似然。即,个体i可能属于某种类型,似然函数
$L_i(\theta_{\tau})$捕捉了该个体数据与类型$\tau$相关的匹配度。
混合似然$\sum_{\tau=1}^{K} \pi_{\tau} L_i(\theta_{\tau})$这表示对所有类型的加权平均,权重是$\pi_{\tau}$,即每种类型的概率。通过这种加权求和,模型允许每个个体属于不同的类型,并通过类型的权重(概率)来加权它们的贡献。

每个个体可能属于不同的类型,这些类型有不同的收入和行为模式。每个数据点i来自某个成分,但成分的归属是未知的,因此需要将每个数据点的似然表示为各成分似然的加权和,然后取对数并求和得到整个样本的对数似然。因此通过混合模型,能够捕捉到个体之间的异质性。


如此,通过最大似然计算,就可以得到各种参数的似然估计。



% subsection 实证模型 (end)

\section{数据上}

地区数据的界限
由于劳动力的迁移在不同省之间呈现趋势 但在迁入大省中仍然存在人口净流出市 这说明劳动力移动的法律边界和现实边界存在重合 但是最好应该以市为标准 当然这一准则不包括某些政策带来的效应
上面说了为什么界限要往下卡在市,下面说一下为什么界限不继续往下到区或者村。首先是政策往往以市为最小执行单位;其次是在市内迁移中个人会为了出于固定资产投资等非模型考量的原因,这些原因并不是因为在迁居的地区能提供更好的个人预期工资
市级包含了县、乡、村等行政级别 每个市都包含了城市与农村(虽然城市化率略有不同)避免了城乡之间的划分对立


人口数据的界限
许召元2007

在我国存在广泛的农民工进城现象 大多数农民工只是暂时居住在


本研究使用CFPS2010年至2022年年两期数据构成面板进行影响评估。CFPS采用的是内隐分层、多阶段、多层次、与人口规模成比例的概率抽样方式,样本覆盖除中国港澳台、新疆、西藏、青海、内蒙古、宁夏和海南之外的其他省/市/自治区。这些地区的人口占全国的95\%左右,因此,CFPS的数据是一个具有全国代表性的高质量数据库。

\textit{关键结果变量为个人的年总收入 (PIncome)。本研究选取的关于个人总收入的变量为“qk601”(2010年) 和“p\_income”(2014 年) 。在面板数据中,两个变量统一命名为“PIncome”。该变量来自 CFPS 问卷的 “K 部分: 个人收入”,具体题目为“您 ( 去年)个人的年总收入是元?”。
CFPS在公布历期数据之前,对收入部分结果进行了修正,以满足各年之间的可比性。控制变量。参考以往研究 (卜茂亮等,2011; 黄国英、谢宇,2017; 谭燕芝等, 2017; 周广肃、孙浦阳,2017) 并结合CFPS数据的可获得性和完整性,本文选取的控制变量包括地域 (所在省份)、常住地 (城镇或农村)、性别、民族、年龄、受教育年限、婚姻状态、自评健康状况、认知能力、非认知能力、对自己未来的自信心和职业等。其中CFPS的非认知能力来自访员在理解能力、配合程度、接人待物水平、回答的可信程度和语言表达能力5个方面对受访者的评价平均得分; 认知能力包括语文和数学方面的测试得分 (黄国英、谢宇,2017)}


基于使用了CFPS从2010年到2022年间的数据
CFPS数据记录了个人的迁移轨迹(将访问当年的地址视作居住地址)
年龄
收入


气候数据:
由于气候和空气质量在短期内发生变化的可能性较小,气候的对比不是局部发生变化的,以及气候数据在劳动力迁移决策中是用于进行横向对比而非纵向对比的,所以本文选用常量对每个省份的空气质量、日照时间、平均气温进行跨时间同一数值赋值。


自然灾害数据、城市人口数据、医疗数据、教育数据等数据来自于国家统计局公布的数据统计年鉴。

\textit{PM2.5 浓度数据来源于 Van Donkelaar et al. ( 2016) 计算的全球年度 PM2. 5 卫星栅格数据。我们 首先利用地理信息系统将每个栅格定位到其空间位置所在的城市上,然后将落在每个城市内的所有 栅格数据进行平均,即可得到各个城市在不同年份的 PM2. 5 浓度水平。与地表监测的点源数据相比, 卫星观测数据在时间和空间上的覆盖范围更广,能够与我们的流动人口数据更好的匹配; 同时,卫星 监测数据相对更加客观和准确,能够避免很多人为因素导致的测量偏误( Ghanem \& Zhang,2014) 。}


地方教育经费、地方医疗经费来自各省地方财政局网站公布的数据。
由于教育的变量存在高度相关性
所以使用主成分分析法将教育变量整合为一个指标
同样的
对于医疗资源使用熵值法合成一个指标


房价数据来自于xxx。com网站

距离数据
各省市以其省会为经纬度源自geopy的Nominatim
广义距离代表了各种形式的距离 比如经济距离

对于文化亲近度
本文将距离、方言分布、文化分布作PCA分析得到文化亲近度
地理亲近度使用指数衰减函数或反距离加权$\text{Closeness}=e^ {-\lambda \cdot Distance}$,(λ为衰减参数)
饮食文化相似度菜系分类树(如八大菜系及其子类),省份所属菜系标签,基于树形结构的最近公共祖先(LCA)距离
\begin{equation}
  \text{Similarity}=\frac{1}{1+\text{LCA depth}}
\end{equation}
方言相似度基于各省方言人口比例(如吴语、粤语、官话等)采用余弦相似度或Jensen-Shannon相似度比较两省方言分布向量
最后使用熵值法得到文化亲近度指标
在\cite{LiuYuYunLaoDongLiKuaFangYanLiuDongDeDaoUXingMoShi2015}的劳动力跨方言流动的倒U型模式中,作者基于《汉语方言大词典》和《中国语言地图集》中的方言分区,计算了全国278个地级市间的方言距离,并与"2012年中国劳动力动态调查数据"相匹配,构建了一个劳动力跨方言流动的微观数据库,验证了理论模型的预测





\section{方法上} % (fold)
\label{sub:方法上}


\subsection{迁移序列的简化}

出于计算内存上的优化,本文简化了迁移序列的记录

\subsection{程序代码设计}
基于以上的推导
在代码中需要先求出个体的似然
而个体的总似然是迁移选择概率和工资观测概率的乘积
那么集体的似然函数同时对未观测的随机效应(固定效应、地区匹配效应等)进行积分(离散求和):
\begin{equation}
  L_{i}=\sum\limits_{\text{所有随机效应组合}}[\prod_{t}\rho(x_{t},j_{t})⋅P(w_{t}|\text{随机效应}) ]
\end{equation}

本文采用混合似然估计法(Mixed Likelihood Estimation)进行动态离散选择模型的参数求取。由于模型涉及高维状态空间与跨期动态关联性,计算过程具有三重复杂性:其一,个体选择行为需通过贝尔曼方程递归求解;其二,状态转移矩阵需处理多重内生性关联;其三,价值函数收敛需满足跨期一致性条件。

传统计量工具(如Stata)受限于矩阵运算效率与迭代算法架构,在应对此类包含200+状态变量、需进行10\^6量级矩阵运算的复杂场景时,存在计算耗时指数级增长与内存溢出的双重瓶颈。
为此,本文构建基于Python的数值计算框架,与传统的Stata相比,Python在实现复杂的动态离散选择模型方面具有显著优势:
首先,Python的开源生态系统提供了丰富的科学计算库,如\lstinline{NumPy}、\lstinline{Pandas}和\lstinline{PyTorch},这些库为处理大规模面板数据和实现复杂的优化算法提供了强大支持。特别是\lstinline{PyTorch}的自动微分功能,使得计算复杂模型的梯度变得简单高效,这在传统Stata环境中难以实现。
其次,Python的面向对象编程范式使得模型的各个组件可以被清晰地封装,提高了代码的可读性和可维护性。在本项目中,我们将模型的不同部分(如个体似然函数\lstinline{IndividualLikelihood}、参数估计器\lstinline{ModelEstimator}等)封装为独立的类,使得整体架构更加清晰。
此外,\lstinline{joblib}库提供的并行计算能力显著提高了计算效率,特别是在处理大量个体似然函数计算时;
本文设计逆向归纳算法(Backward Induction),通过倒序年龄迭代(从T期至$t_0$期)动态求解各状态节点的期望价值函数(EV矩阵),确保跨期决策路径的全局最优;
并运用SciPy优化器进行参数空间搜索,通过蒙特卡洛模拟生成选择概率曲面,最终实现模型参数在95\%置信区间内的有效估计。


对于求取参数的方法,传统上使用Newton线搜索方法上求解似然函数最大值,虽然收敛速度快,但计算海塞矩阵需要消耗大量资源。
虽然可以通过对似然函数进行LU分解,在计算海塞矩阵的逆时提供可靠帮助,降低数值不稳定性,
但计算海森矩阵的代价仍然较高,尤其是在参数较多时。
本文做出改进,
采取以自动微分(\lstinline{PyTorch})为核心,使用现代优化算法Quasi-Newton方法中的L-BFGS方法避免收敛困难和数值不稳定,并利用离散化与并行处理提升效率。
BFGS是一种拟牛顿法,通过迭代逼近目标函数的海塞矩阵的逆矩阵,从而避免直接计算二阶导数。它只需要利用梯度信息来更新海森矩阵的近似,使得每次迭代都能更准确地找到下降方向。在高维参数空间中,避免了直接计算海森矩阵的高计算成本。
在具体的梯度计算中,自动微分通过计算图追踪运算过程,利用链式法则自动计算导数,避免数值微分的截断误差。当模型复杂、参数众多,且需要频繁计算梯度时(如神经网络训练),自动微分显著优于手动编码梯度。对于需要高阶导数或大规模并行计算的情况尤为适合。
具体而言,
本文使用类方法分装待估参数
类继承
\lstinline{torch.nn.Module},所有参数为\lstinline{torch.nn.Parameter},支持自动梯度计算。

本文对于固定效应和地区匹配效应upsilon等支撑点离散化方法的处理与原作者的做法保持一致,
通过网格遍历求和。
对于已经用支撑点离散近似后的参数,在进行网格搜索时,
\lstinline{joblib.Parallel}支持将数据存储在磁盘上的内存映射文件中,这样多个进程可以共享同一份数据,
而无需每个进程都复制一份,从而显著降低内存消耗并提升 I/O 效率。

本文的优化方法基于深度学习,
模型估计结果对初始值存在一定敏感性,
初始值的选择可能会影响收敛速度和结果。例如,如果真实值接近-0.1,好的初始值能加快收敛。
虽然L-BFGS优化器在一定程度上缓解了这一问题,但仍需根据经济学理论进行初始值赋予能促进优化方法的应用,例如距离增加可能降低迁移概率,因此\lstinline{gamma_distance}的预期应该为负,将其初始值设为负符合理论预期,有助于引导优化方向。




% ---------------------------------------- 估计结果 ----------------------------------------
\chapter{估计结果}

本文尝试回答一下几个问题:
\begin{itemize}
  \item 影响劳动力流动的因素是什么?到底哪些因素是负面因素(迁移摩擦)?
  \item 不同人群是否面对不同的劳动力迁移摩擦阻力?
  \item 本地的福利待遇对于迁移是否显著?这意味着是否可以通过政府补贴购买人才
  \item 
\end{itemize}


\cite{XiaYiRanChengShiJianDeMengMuSanQianGongGongFuWuYingXiangLaoDongLiLiuXiangDeJingYanYanJiu2015} 利用 2005 年 1\%人口抽样调查中劳动力流动的微观数据与 220 个地级市的城市特征 数据研究发现公共服务对劳动力流入产 生 吸 引 力,同 时 房 价 对 劳 动 力 流 入 也 有 正 向 作 用 ,他 们 认 为这是因为房价“资本化”了部分未观察到的公共服务或城市特征,此文关注点不在房价,而且使 用的普查 数 据 没 有 覆 盖 近 十 年 来 中 国 房 价 暴 涨 的 阶 段。

\section{基准回归} % (fold)
\label{sub:基准回归}



% subsection 基准回归 (end)
\section{不同学历人群的差别} % (fold)
\label{sub:不同学历人群的差别}



% subsection 不同学历人群的差别 (end)

% ---------------------------------------- 结论 ----------------------------------------
\chapter{结论与展望}

当然以城乡二元对立为代表的思想在我国依旧有非常重要的应用 因为我国依旧有大量依附于城乡关系的社会体系、福利体系等种种重要的制度
甚至在研究二元对立话题中依旧可以引入例如Rosen Roback这样的经典模型
例如 
\cite{GuoDongMeiChengXiangRongHeDeShouRuHeFuLiXiaoYingYanJiuJiYuYaoSuPeiZhiDeShiJiao2023}指出城乡融合的收入和福利效应研究——基于要素配置的视角
但对于在破除劳动力迁移摩擦、开放劳动力要素自由流动的当下
抛开这种二元对立的思想是越来越重要的
这也自然而然地引出了空间均衡与最有选址两种思路



由于普遍偏好事少离家近的特征,政策可以针对性地对周边省份进行补贴
相反 对于十分遥远地区的人力资源 即使补贴了显性的迁移成本 也存在较长的文化、心里距离 所以在边际上可能并不值得投入



本文可以改进的地方:

添加约束

添加宏观变量

使模型作为微观基础从而宏观化
\textit{近年来,一些研究将上述两种基础模型结合起来,在空间均衡模型中加入了微观层面的动态迁移决策特征。这些动态一般均衡迁移模型明确考虑了空间工资差异的来源及其对净迁移和总迁移的影响,并允许存在不同类型的空间障碍,如劳动力重新配置摩擦和信息摩擦。这类模型特别关注迁移如何作为调节各地劳动力市场长期均衡的机制。例如,Coen-Pirani(2010)开发了一个动态一般均衡模型,强调了个体迁移决策中的不可观测异质性。该模型刻画了总迁移流动和净迁移流动的共同模式,其中前者由个体匹配的偶然冲击驱动,后者由持续的生产率冲击驱动。工人会迁移到正受到生产率冲击的地区,并在迁移后发现其偶然匹配的质量。新迁移的工人比长期居住者更可能继续迁移,因为后者选择留在某地是由于他们已经获得了相对较好的匹配。该校准模型可以解释为何人口流入较多的地区往往也伴随着较多的流出,这一现象在仅研究净流入的模型中无法得到解释。结合偶然匹配效应的空间均衡模型还能够解释新迁入工人与迁出工人在年龄、教育和行业等方面的相似性,这些特征无法仅通过个体位置选择模型或仅基于可观测工人异质性或地点特定冲击的模型来解释。}

从而引入其他变量

以家庭为基本单位


% ---------------------------------------- 附录 ----------------------------------------
\newpage
\appendix

\chapter{McFadden条件概率}

个体选择选项j的条件是其总效用最大,即
$u_j + \zeta_j > u_k + \zeta_k, \forall k \neq j$。
这可以转化为
$\zeta_j - \zeta_k > u_k - u_j, \forall k \neq j$。
假设误差项$\zeta_j$独立且服从相同的极值分布,则对于每个$k \neq j$,有
$\Pr(\zeta_j > \zeta_k + (u_k - u_j)) = \frac{1}{1 + \exp(u_k - u_j)}$。
在RUM模型中,当误差项独立时,选择j的概率为所有选项k的独立事件同时发生的概率,即
$P_j = \prod\limits_{k \neq j} \Pr(\zeta_j > \zeta_k + (u_k - u_j))$。
代入单个事件的概率表达式,选择j的概率是所有$k \neq j$的条件概率的乘积
$P_j = \prod\limits_{k \neq j} \frac{1}{1 + \exp(u_k - u_j)}$。
对$P_j$取对数
$\ln P_j = - \sum_{k \neq j} \ln(1 + \exp(u_k - u_j))$。
通过指数函数的性质,将上述表达式转换为
$P_j = \frac{\exp(u_j)}{\sum\limits_{k \in C} \exp(u_k)}$。

\chapter{效用等价于收入的特殊情况}
假设效用函数的线性形式为$U(x)=a X$,其中a为边际效用参数。居民可在固定利率$r$下无限制借贷,预算约束满足消费现值等于收入现值。
居民可以留在原地(永久收入为$Y_h$)或迁移(永久收入为$Y_p$,后者需支付一次性迁移成本 $M$。
假设迁移成本 
$M$
为即期支出,且永久收入为永续年金。

若居民留在原地,其永久收入的现值为
$W_h = \sum\limits_{t=0}^\infty \frac{Y_h}{(1+r)^t}=\frac{Y_h}{r}$,
总效用为$U_h=a W_h = \frac{a Y_h}{r}$。
若居民选择迁移,需支付即期成本 
$M$
,迁移后的永久收入现值为
$W_h = \sum\limits_{t=0}^\infty \frac{Y_p}{(1+r)^t}=\frac{Y_p}{r}$,
由于迁移成本 
$M$
为即期支出,净现值为
$\mathcal{W}=W_p-M=\frac{Y_p}{r}-M$,
总效用为
$U_p=a(\frac{Y_p}{r}-M)$
迁移的条件为$U_p>U_h \Rightarrow a(\frac{Y_p}{r}-M) > a \frac{ Y_h}{r} \Rightarrow Y_p-Y_h > rM$。
期望效用最大化等价于选择净现值更高的选项,即$\max{a(\frac{Y_p}{r}-M), a \frac{ Y_h}{r}}$。

\chapter{Rust极值分布}
公式中的随机效用项假设服从于一类极值分布
We assume that $\zeta_j$ is drawn from the Type I extreme value distribution. In this case, using arguments due to McFadden (1973) and Rust (1987), we have
$$\exp\left(\bar{v}(x)\right) = \sum_{k=1}^J \exp\left(v(x, k)\right)$$

这表示如果变量服从一类极值分布,那么$\exp\left(\bar{v}(x)\right)$可以表示为所有$v(x, k)$的指数和
它意味着,在状态x 下,选择某一选项j的概率与效用的指数值成比例。
这个性质广泛用于预测个体选择的分布。

\chapter{地点选择概率}
$\rho(x,j)=\frac{\exp(v(x,j))}{\sum\limits_{k=1}^{J} exp(x,k)}$

Probability that a person in state $x$ will choose location $j$ can then be written as
$$\rho(x,j)=\exp[v(x,j)-\bar v(x)]$$
% section 附录_证明 (end)证明

\chapter{log} % (fold)

令$\mathcal{K}_{it}=(\mathcal{K}_{it}^{0},\mathcal{K}_{it}^{1})$表示当前位置和上一个位置

已知\ref{eq:地点选择概率},
个体i在t时期选择目的地的似然概率是$\lambda_{it}(\omega^{i},\theta_{\tau})$
同时在
\begin{equation}
  \lambda_{it}(\omega^{i},\theta_{\tau})=\rho(x,j)=\rho_{h(i)}(\ell(i,t),\omega_{\nu}^{i}(\mathcal{K}_{it}^{0}),\omega_{\nu}^{i}(\mathcal{K}_{it}^{1}),\omega_{\xi}^{i}(\mathcal{K}_{it}^{0}),\omega_{\xi}^{i}(\mathcal{K}_{it}^{0}),a_{it},\ell^{0}(i,t+1),\theta_{\tau})
\end{equation}
公式表明个体的选择不仅受到过去历史(如上一个地点和上一个选择)和随机效应(如匹配效应和暂态效应)的影响,还考虑了个体的长期偏好和未来选择的影响。
通过对这个似然函数进行建模,我们可以估计个体在特定情况下做出选择的概率,从而理解个体在复杂环境下如何做出决策。


由于暂态效应服从期望为$0$的正态分布,并且已知经济收益函数设定为公式\ref{eq:经济收益函数},
令$\Psi$表示标准正态分布的概率累积函数,$\phi$表示标准正态分布的概率密度函数,可知经济收益函数的密度函数为
\begin{equation}
  \Psi_{it}(\omega^{i},\theta)=\phi(\frac{w_{it} - \mu_{\ell^{0}(i,t)}-G(X_{i},a_{it},\theta)-\nu(\omega_{nu}^{i}(\mathcal{K}_{it}^{0}))-\eta(\omega_{\eta}^{i})  }{\sigma_{\epsilon}(\omega_{\epsilon}^{i})})
\end{equation}
其中$\mu_{\ell^{0}(i,t)}$:是个体i在位置$\ell$基础工资水平;$G(X_i, a_{it}, \theta)$:代表个体$i$在时间$t$受到的外部影响(如经济环境、工作特征等);$\nu(\omega_{\nu}^i(\mathcal{K}_{it}^0))$、$\eta(\omega_{\eta}^i)$:分别是地区匹配效应和固定效应,它们是收入的随机组成部分,影响个体的收入;$\sigma_{\epsilon}(\omega_{\epsilon}^i)$:表示暂态效应的标准差,它反映了短期收入波动的程度。

这是在时期$t$给定个体$i$的选择和随机效应(如匹配效应、固定效应等),观察到收入$w_{it}$的概率概率,即被观测收入的似然概率。

\newpage
\bibliography{Papers}
\bibliographystyle{gbt7714-author-year}
\end{document}
